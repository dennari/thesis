Sequential data and timeseries, causal relations, consequences of actions

State and measurement and modeling

Parameters, when and why are they interesting?

Bayesian approach, priors, probability distributions, uncertainty





 

\parencite{Murphy2002}
\todo{SSMs vs Box-Jenkins}
\todo{Role of static parameters}
\todo{Importance of estimating static parameters}
\todo{Overview of different approaches}
\subsubsection*{Example: Ballistic object}
Suppose we are observing an airborne missile in two dimensions.
We obtain a sequence of measurements with an unspecified instrument,
reporting the noisy location of the missile. We assume the situation
is governed by Newton's dynamics and we would like to estimate the weight
of the missile.

Newton's laws are specified in continous time, so that we use a simple
Euler integration scheme in order to model the situation with a discrete-time 
linear-Gaussian SSM. For simplicity, it is assumed that the only force affecting
the missile is gravitation. Then
\begin{align}
	\dpd[2]{\y(t)}{t} &= \frac{\f}{m}\\
	\y &= \bm{x & y}^\tr\\
	\f &= \bm{f_x & f_y}^\tr	
\end{align}

 
\begin{figure}[htp]
\begin{center}
	\missingfigure{A missing 1D RW simulation}
  \caption{Simulation from the 1D RW model}
  \label{fig:rw1d}
\end{center}
\end{figure}
