Sequential data and timeseries, causal relations, consequences of actions

Modelling temporal data is of fundamental interest in many branches of science.
This is understandable, since natural organisms inhabit a dynamic environment and
it is natural organisms, such as ourselves, that easily draw our attention and raise questions
deserving a scientific answer. The accuracy of the predictions, based on earlier observations,
is indeed a property that can easily affect the survival of the forecaster.  

Causality and consequences of actions are concepts tied to the passage of time.
Thus it is often time, that dictates the natural ordering of the datapoints
in sequential data. It is however not necessarily so, and any other single
physical dimension may also be used.

Commonly the sequential data is a result of making \emph{measurements} on a
\emph{system} of interest. In order to answer questions quantitatively,
the system and the measurements should be mathematically \emph{modelled}. The class of mathematical
models for dynamical systems we will be concerned with are called \emph{state space models} (SSMs).
Intuitively, SSMs make a clear distinction between the system and the measurements. At any instant,
the system is at a certain \emph{state}. In general, it is this state and its evolution that we are interested in.
However the state is \emph{hidden} (or \emph{latent}) and the inference on the state has to be made entirely
based on the measurements.  Often at least part of the state is conceptually part of the measurements,
but even this part of the state is still latent, since the measurements are always assumed to be \emph{noisy}.
To give a simple but often used example, let us consider the target tracking problem. In this case the state
could include the position, velocity and acceleration of the target and our measurements could consists
of a sequence of noisy angular readings between the line of sight and a reference line. This situation is
known as \emph{bearings-only target tracking}.

The noisiness forces us to assume a probabilistic framework. In this thesis the viewpoint
is decidedly \emph{Bayesian}. In Bayesianism, the complete answer
is always the \emph{posterior probability distribution}. Thus instead of answering
with a single value or a value with error bounds, the answer is the probability
density function of the interesting quantity given data.

State and measurement and modeling

Markov chain

Parameters, when and why are they interesting?

Bayesian approach, priors, probability distributions, uncertainty

Linear Gaussian 
  


Nonlinear Gaussian
  Bearings-only target tracking
  Stochastic volatility



 

\parencite{Murphy2002}
\todo{SSMs vs Box-Jenkins}
\todo{Role of static parameters}
\todo{Importance of estimating static parameters}
\todo{Overview of different approaches}
\subsubsection*{Example: Ballistic object}
Suppose we are observing an airborne missile in two dimensions.
We obtain a sequence of measurements with an unspecified instrument,
reporting the noisy location of the missile. We assume the situation
is governed by Newton's dynamics and we would like to estimate the weight
of the missile.

Newton's laws are specified in continous time, so that we use a simple
Euler integration scheme in order to model the situation with a discrete-time 
linear-Gaussian SSM. For simplicity, it is assumed that the only force affecting
the missile is gravitation. Then
\begin{align}
	\dpd[2]{\y(t)}{t} &= \frac{\f}{m}\\
	\y &= \bm{x & y}^\tr\\
	\f &= \bm{f_x & f_y}^\tr	
\end{align}

 
\begin{figure}[htp]
\begin{center}
	\missingfigure{A missing 1D RW simulation}
  \caption{Simulation from the 1D RW model}
  \label{fig:rw1d}
\end{center}
\end{figure}
