Arguably some of the most interesting phenomena are dynamic in nature. 
Our own bodies are a good example. A widely used way of obtaining information
about a physical entity is taking \emph{measurements}. If there is reason
to suppose that the entity is changing with respect to the quantity
we are measuring, it is a good idea to take multiple measurements over
a time-period. The result is a discrete-time signal or a \emph{time series}.
It should be emphasized, that one almost always measures at discrete time-instants,
while the object we are measuring inhibits the world of continous-time.
It seems obvious that one should be careful in discerning the measurements,
which have limited capacity of transmitting information, from the target of inquiry.
How should we connect what we can measure to what we suppose, a priori, we are
measuring? This is the domain of modeling.

A very 


Signals are by definition 

The human body manages an intricate network of systems in order to remain
stable, in homeostasis. This steady state, which includes control of such quantities
as temperature and energy balance, is essential for the correct functioning of any one of us.
Deviation from homeostasis can be observed for instance by observing
\emph{signals}, i.e. measurable quantitites which change as a function of time.
%These biosignals include for example the electric field induced by the heart or the
 


\parencite{Murphy2002}
\todo{SSMs vs Box-Jenkins}
\todo{Role of static parameters}
\todo{Importance of estimating static parameters}
\todo{Overview of different approaches}
\subsubsection*{Example: Ballistic object}
Suppose we are observing an airborne missile in two dimensions.
We obtain a sequence of measurements with an unspecified instrument,
reporting the noisy location of the missile. We assume the situation
is governed by Newton's dynamics and we would like to estimate the weight
of the missile.

Newton's laws are specified in continous time, so that we use a simple
Euler integration scheme in order to model the situation with a discrete-time 
linear-Gaussian SSM. For simplicity, it is assumed that the only force affecting
the missile is gravitation. Then
\begin{align}
	\dpd[2]{\y(t)}{t} &= \frac{\f}{m}\\
	\y &= \bm{x & y}^\tr\\
	\f &= \bm{f_x & f_y}^\tr	
\end{align}

 
\begin{figure}[htp]
\begin{center}
	\missingfigure{A missing 1D RW simulation}
  \caption{Simulation from the 1D RW model}
  \label{fig:rw1d}
\end{center}
\end{figure}
