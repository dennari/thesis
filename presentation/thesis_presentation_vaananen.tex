%%% Document class beamer, do not change
\documentclass[t]{beamer}
%% Optional packages
\usepackage[utf8]{inputenc}
%% Required if you are writing in other language than English
%\usepackage[finnish,swedish,english]{babel}
\usepackage{listings}
\usepackage{subcaption}


%% Input the document information, note the use of short information in the footer
% The document title. Shows an example how linebreaks can be obtained.
\title[SSM Parameter. est.]{%
Gaussian filtering and smoothing based parameter estimation in nonlinear models for sequential data
}

% The subtitle, e.g. the conference or course name.
% An abbreviation (or similar) in the short version is handy.
\subtitle[Master's thesis presentation]{%
Master's thesis presentation
}



% The author names. Shows a second example of linebreaks.
% Note the use of \inst, see the \institute command!
\author[Väänänen]{%
	Ville Väänänen\\
	{\scriptsize\url{ville.vaananen@aalto.fi}}
}
% The authors' affiliations.
\institute[Aalto University School of Science]{%
	Department of Biomedical Engineering and Computational Science\\
	Aalto University School of Science
}
% The date. Default date is \today.
%\date[Short Example Date]{%
%	Long Example Date, Default Date is \textbackslash today%
%}
% The subject. This is stored only in the PDF information.
\subject{Document Subject Example}

%\usepackage{mylayout}

% Theme loading
\usepackage[SCI]{aaltologo}
\usetheme{Aalto}

%%% Begin the document
\begin{document}

%%% This file contains the code for the sample0x.tex files.

%% Create the title page
\maketitle

\begin{frame}
	\frametitle{Outline}
	\tableofcontents
	% You might wish to add the option [pausesections]
\end{frame}

\section{State space models}

\begin{frame}
	\frametitle{Data}
	\vskip -15pt
	\begin{itemize}
	  \item adfasdf
	\end{itemize}
\end{frame}
	
	
%\frame[plain]{
%	\vskip -25pt
%}


\section{Geographical data in R}


\section{Preparing the observation window}

\begin{frame}
	\frametitle{The observation window}
	\vskip -15pt
	\begin{itemize}
  \item From a map (shapefile with lines) to polygon 
  \item A huge effort
  \item Needed for the mesh 
\end{itemize}
\end{frame}






\end{document}