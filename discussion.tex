\subsection{Comparisons}
Both the direct method of section~\ref{sec:grad} and the EM algorithm of section~\ref{sec:EM_SSM}
have their strengths and weaknesses. Neither of them can be said to eclipse the other in an absolute sense.
In this section we will go through some features of the said algorithms found in the literature.
Then a more detailed analysis will be performed on two essential aspects of any estimation algorithm:
the convergence properties, i.e. when should one expect the algorithm to find a local maximum and
the computational requirements of the algorithms.

\textcite{Cappe2005} contains a list of arguments in favor of either of the methods. The following list includes
those and additional points with comments:
\begin{description}
  \item[Direct]\hfill
\begin{itemize}
  \item\emph{No smoother needed} The log-likelihood, or an approximation to it, can be evaluated
  with forward-filtering
  \item\emph{No M-step}. There is no need to figure out model-dependent maximization
  formulas even in nonlinear models 
  \item\emph{Faster convergence}. Advanced gradient-based optimization
  methods can reach convergence speeds that are close to quadratic
\end{itemize}
  \item[EM]\hfill
  \begin{itemize}
  \item \emph{Simple to implement}. This argument is often put forward in favor of the EM
 algorithm. However in practice when using the gradient-based method one would use any one of
the off-the-self nonlinear optimizers and not re-implement one. Thus which one is easier to implement
boils down to gradient computation. If the model is linear or linear-in-the-parameters the EM
algorithm doesn't need any gradient information.
  \item\emph{Parameter constraints}. In \textcite{Cappe2005} it is argued that
since the M-step maximization equations are so simple, including parameter constraints
are easier in the EM algorithm. This again depends on if the model is linear or linear-in-the-parameters.
  \item\emph{Parameterization independent}. This again depends on if one has to use gradient-based
 optimization in the M-step. If not, then the EM algorithm is parameterization independent. In the gradient-based
 method the gradient and the Hessian, and so the convergence, are affected by the parameterization.
\end{itemize} 
\end{description}
\todo{shortcomings of Gaussian filtering}
\todo{Dual and joint filtering}
\todo{particle filtering}
\parencite{Haykin2001}
\parencite{Kantas2009,doucet2001sequential,Gordon1993,Lindsten2010}
