
In this section we are concerned with finding, exact if possible and approximate
otherwise, filtering and smoothing
distributions of Equations \eqref{eq:filt_bayes} and \eqref{eq:smooth_bayes}.
In fact we will focus on the somewhat more general problem for deriving
the cross-timestep joint pdfs. In state estimation it is assumed that the parameter $\Th$ is given.
Thus throughout this section we will, in general, suppress the dependence on $\Th$ and use the shorthands specified in \eqref{eq:shorthand_theta}. 

\subsection{Linear-Gaussian State Space Models}

A linear map $Q:\mathcal{A}\to\mathcal{B}$ satisfies the equation
\begin{align}
	Q(\alpha\v{a}+\beta\v{b}) &= \alpha Q(\v{a})+\beta Q(\v{b}), \quad \forall\;\; \v{a},\v{b}\in\mathcal{A}\;\;\&\;\;\alpha,\beta \in \R.
\end{align}%
%
Since linear maps can be described by matrices, stationary linear-Gaussian SSMs 
are described by the subset of SSMs of the form \eqref{eq:ssm_general} where  
\begin{align}
	\ff&=\v{A}(\Th)\x_{k-1}\\
	\hh&=\v{H}(\Th)\xk.
\end{align}
The $d_x\times d_x$ matrix $\v{A}\equiv\v{A}(\Th)$ is called the \emph{transition matrix} and
the $d_y\times d_x$ matrix $\v{H}\equiv \v{H}(\Th)$ the \emph{measurement matrix}.
These linear-Gaussian SSMs, equivalently known as \emph{linear dynamical systems} \parencite{Bishop2006},
are one of the few cases where computing the predictive, filtering and
smoothing distributions is tractable \parencite[see, e.g.,][]{Sarkka2006}. As can be seen,
all of the aforementioned distributions stay Gaussian.

\subsubsection{Kalman filter}

The Kalman filter is the best known filter, first presented in
the seminal article of \textcite{Kalman1960}. It provides the
closed form solution to computing the predictive and filtering distributions
of Equations \eqref{eq:pred_bayes} and \eqref{eq:filt_bayes}.
With the help of Lemmas \ref{lemma:gaussian_joint} and \ref{lemma:gaussian_cond},
deriving the Kalman filter equations is quite straightforward \parencite{Sarkka2006}.
The resulting recursions are \parencite{jazwinski1970stochastic}:
\begin{subequations}
\label{eq:Kalman_filter}
\begin{description}
\addtolength{\leftskip}{1cm}
\item[Predict:]
\begin{align}
	\v{m}_{k|k-1}&=\v{A}\v{m}_{k-1|k-1}\\
	\v{P}_{k|k-1}&=\v{A}\v{P}_{k-1|k-1}\v{A}^\tr+\v{Q}
\end{align}
\item[Update:]
\begin{align}
	\v{v}_k&=\v{y}_k-\v{H}\v{m}_{k|k-1}\\
	\v{S}_k&=\v{H}\v{P}_{k|k-1}\v{H}^\tr+\v{R}\\
	\v{K}_k&=\v{P}_{k|k-1}\v{H}^\tr\v{S}_{k}^{-1}\\
	\v{m}_{k|k}&=\v{m}_{k|k-1}+\v{K}_k\v{v}_k\\
	\v{P}_{k|k}&=\v{P}_{k|k-1}-\v{K}_k\v{S}_k\v{K}_{k}^\tr
\end{align}
\end{description}
\end{subequations}
This includes the sufficient statistics for the $T$
joint distributions 
\begin{align}
	\Pdf{\v{x}_k,\v{y}_k}{\v{y}_{1:k-1},\gv{\theta}}
	&=\N[
	\begin{bmatrix}
		\v{x}_k\\\v{y}_{k}
	\end{bmatrix}
	]{
	\begin{bmatrix}
		\v{m}_{k|k-1}\\
		\v{H}\v{m}_{k|k-1}
	\end{bmatrix}
	}{
	\begin{bmatrix}
		\v{P}_{k|k-1} & \v{P}_{k|k-1}\v{H}^\tr\\
		\v{H}\v{P}_{k|k-1}^\tr & \v{S}_k  
	\end{bmatrix}
	}
	\label{eq:joint_per_kalmanstep}
\end{align}

\subsubsection{Rauch-Tung-Striebel Smoother}

Once the filtering distributions are obtained going \emph{forward} in time,
the joint smoothing dustributions \eqref{eq:smooth_joint_bayes} can be computed
going \emph{backwards} in time. In this computing order sense, the last filtering distribution
is the first smoothing distribution. In the linear-Guassian case,
the Rauch-Tung-Striebel (RTS) smoother gives the statistics $\v{m}_{k|T}$ and 
$\v{P}_{k|T}$ \parencite{jazwinski1970stochastic,Rauch1965} in Equation \eqref{eq:smoothing_gaussian}.
We will use a version that gives the joint distribution of the states across a timestep, since
the cross-timestep covariance will be needed in the parameter estimation phase.
Assuming now that all the predictive and filtering distributions, i.e $\N[\x_{k+1}]{\m_{k+1|k}}{\v{P}_{k+1|k}}$ and $\N[\xk]{\m_{k|k}}{\v{P}_{k|k}}$ respectively,
are available, the RTS recursions can be written as
\begin{subequations}
\begin{align}
	\v{J}_k&=\v{P}_{k|k}\v{A}^\tr\v{P}_{k|k+1}^{-1}\\
	\v{m}_{k|T}&=\v{m}_{k|k}+\v{J}_k\left(\v{m}_{k+1|T}-\v{m}_{k+1|k}\right)\\
	\v{P}_{k|T}&=\v{P}_{k|k}+\v{J}_k\left(\v{P}_{k+1|T}-\v{P}_{k+1|k}\right)\v{J}_k^\tr
	%\v{C}_{k|T}&=\v{P}_{k|k}\v{J}_{k-1}^\tr+\v{J}_k\left(\v{C}_{k+1|T}-\v{A}\v{P}_{k|k}\right)\v{J}_{k-1}^\tr \label{eq:rts_cross_timestep_covariance}
\end{align}
\end{subequations}
This includes the sufficient statistics for the $T$
joint distributions 
\begin{align}
\begin{split} 
	\Pdf{\v{x}_k, \v{x}_{k-1}}{\v{Y},\Th}&=
	\N[\bm{\v{x}_k\\\v{x}_{k-1}}]{
	\bm{
		\v{m}_{k|T}\\
		\v{m}_{k-1|T}
	}
	}{
	\bm{
		\v{P}_{k|T} & \v{P}_{k|T}\v{J}_k^\tr\\
		\v{J}_k\v{P}_{k|T} & \v{P}_{k-1|T}  
	}
	}
\end{split}
\label{eq:sum_expectations}
\end{align}


%%%%%%%%%%%%%%%%%%%%%%%%%%%%%%%%%%%%%%%%%
\subsection{Nonlinear-Gaussian SSMs}%%%%%
%%%%%%%%%%%%%%%%%%%%%%%%%%%%%%%%%%%%%%%%%
\label{sec:nonlinear_state}
In the nonlinear case at least one of the mappings $\v{f}$ and $\v{h}$ in
\eqref{eq:ssm_general} is nonlinear (in $\x$). Unfortunately in this case computing the filtering
distributions in closed form becomes intractable and one has to resort to 
some sort of approximations. We can
divide these approximate filtering (and smoothing) solutions into two 
categories \parencite[see, e.g.,][]{Arasaratnam2009}: 
\begin{enumerate}[i)] \addtolength{\leftskip}{.5cm} \itemsep1pt \parskip0pt \parsep0pt
  \item \emph{Local approaches} assume the parametric form of the posterior
  distributions \eqref{eq:pred_bayes}, \eqref{eq:filt_bayes} and \eqref{eq:smooth_bayes} \emph{a priori}. 
  These  methods are analytically inexact but less computationally demanding. This is the category that
we will be concerned with in this thesis. 
  \item \emph{Global approaches} require the use of particle filtering (or sequential Monte Carlo), 
  which is asymptotically exact but computationally demanding
\end{enumerate}%
%
The number of different methods in the first category is substantial,
but a large proportion can be analyzed under the framework of
\emph{Gaussian filtering} (or assumed density filtering
with a Gaussian assumption). As will be shown later, the specific Gaussian filtering 
methods only differ in their chosen numerical integration methods.  

\subsubsection{Gaussian filtering and smoothing}

One approach to forming Gaussian approximations is to assume a Gaussian
probability distribution a priori \parencite{Kushner1967,Ito2000,Wu2006,Sarkka2010}. 
Since a Gaussian distribution is 
defined by its first two moments, a moment matched approximation
can be obtained if the first two moments of the actual probability
distribution can be computed \parencite{Ito2000,Sarkka2006}. As will
be seen, computing these approximations reduces to the problem
of computing multidimensional moment integrals of the form 
\emph{nonlinear function} $\times$ \emph{Gaussian}. 

We shall next derive 
the general form of the three moment integrals and then show how they can be applied
in the specific case of approximating the joint smoothing distribution
of Equation \eqref{eq:smooth_joint_bayes}. Suppose now that  
\begin{align}
	\Pdf{\x}&= \N[\x]{\m}{\P} \label{eq:pa}\\
	\Pdf{\y}{\x}&= \N[\y]{\F*{f}{\x}}{\v{R}}. \label{eq:pb_given_a}
\end{align}
Then $\Pdf{\x,\y}$ is Gaussian only if $\F*{f}{\x}$ is a linear map (with a possible affine constant). Assuming that's not the case,
let us denote a Gaussian approximation
to $\Pdf{\x,\y}$ with
\begin{align}
	\label{eq:joint_gaussian_approx}
	\Pdf{\begin{bmatrix}
		\x\\
		\y
	\end{bmatrix}}&\approx
	\N{
	\begin{bmatrix}
		\gv{\mu}_x\\
		\gv{\mu}_y
	\end{bmatrix}
	}{
	\begin{bmatrix}
		\gv{\Sigma}_{xx}&\gv{\Sigma}_{xy}\\
		\gv{\Sigma}_{xy}^\tr&\gv{\Sigma}_{yy}
	\end{bmatrix}
	}
\end{align}
Then according to Lemmas \eqref{lemma:gaussian_joint} and \eqref{lemma:gaussian_cond}, 
for the moment matched approximation we have to have
\begin{align}
	\gv{\mu}_x&=\v{m}\\
	\gv{\Sigma}_{xx}&=\v{P}\\
	\gv{\mu}_y&=\defint{}{}{\F*{f}{\x}\N[\x]{\m}{\P}}{\x} \label{eq:mean_int}\\
	\gv{\Sigma}_{yy}&=\defint{}{}{\fparen*{\F*{f}{\x}-\gv{\mu}_y}\fparen*{\F*{f}{\x}-\gv{\mu}_y}^\tr\N[\x]{\m}{\P}}{\x}+\v{R}\label{eq:var_int}\\
	\gv{\Sigma}_{xy}&=\defint{}{}{\fparen[\big]{\x-\v{m}}\fparen*{\F*{f}{\x}-\gv{\mu}_y}^\tr\N[\x]{\m}{\P}}{\x}\label{eq:cross_cov_int}
\end{align}

\subsubsection*{Prediction step} 
Since the Gaussian approximation to
\eqref{eq:smooth_joint_bayes} will be calculated by forward
(filtering) and backward (smoothing) recursions, let us assume that we already
have available the filtering distribution of the previous step
\begin{align}
	\Pdf{\xkk}{\y_{1:k-1}} &\approx \N[\xkk]{\m_{k-1|k-1}}{\P_{k-1|k-1}}.
\end{align}
Then
\begin{align}
	\Pdf{\xkk,\xk}{\y_{1:k-1}}&\approx\N[\xkk]{\m_{k-1|k-1}}{\P_{k-1|k-1}}\N[\xk]{\f_{k-1}}{\v{Q}}\\
	&\approx
	\N[\bm{\xkk \\ \xk}]{
	\begin{bmatrix}
		\v{m}_{k-1|k-1}\\
		\v{m}_{k|k-1}
	\end{bmatrix}}{
	\begin{bmatrix}
		\v{P}_{k-1|k-1}&\v{C}_{k-1,k}\\
		\v{C}_{k-1,k}^\tr&\v{P}_{k|k-1}
	\end{bmatrix}
	}
	\label{eq:joint_predictive_approximation}
\end{align}
where by application of Equations \eqref{eq:mean_int}, \eqref{eq:var_int} and \eqref{eq:cross_cov_int} 
\begin{align}
	\begin{split}
	\v{m}_{k|k-1}&=\defint{}{}{\f_{k-1}\N[\xkk]{\m_{k-1|k-1}}{\P_{k-1|k-1}}}{\xkk}\label{eq:prediction_mean_intergral}
	\end{split}\\
	\begin{split}
	\v{P}_{k|k-1}&=\defint{}{}{\fparen*{\f_{k-1}-\v{m}_{k|k-1}}\fparen*{\f_{k-1}-\m_{k|k-1}}^\tr\\
	&\qquad\times\N[\xkk]{\m_{k-1|k-1}}{\P_{k-1|k-1}}}{\xkk}+\v{Q}\label{eq:prediction_variance_intergral}
	\end{split}\\
	\begin{split}
		\v{C}_{k-1,k}&=\defint{}{}{\fparen*{\xkk-\m_{k-1|k-1}}\fparen*{\f_{k-1}-\v{m}_{k|k-1}}^\tr\\
		&\qquad\times\N{\xkk}{\m_{k-1|T}}{\P_{k-1|T}}}{\xkk}\label{eq:prediction_cov_intergral}
	\end{split}
\end{align}%
%

\subsubsection*{Update step}

For the update step we first approximate
\begin{align}
\begin{split}
	\Pdf{\xk,\yk}{\y_{1:k-1}}=\Pdf{\yk}{\xk}\Pdf{\xk}{\y_{1:k-1}}&\approx\N[\yk]{\h_k}{\v{R}}\N[\xk]{\m_{k|k-1}}{\P_{k|k-1}}\\
	&\approx 
	\N[\bm{\xk\\\yk}]{
	\begin{bmatrix}
		\m_{k|k-1}\\
		\gv{\mu}_{k}
	\end{bmatrix}}{
	\begin{bmatrix}
		\v{P}_{k|k-1}&\v{C}_{k}\\
		\v{C}_{k}^\tr&\v{S}_{k}
	\end{bmatrix}
	}.
	\label{eq:joint_update_approximation}
\end{split}
\end{align}
Applying Equations \eqref{eq:mean_int}, \eqref{eq:var_int} and \eqref{eq:cross_cov_int} again,
we get
\begin{align}
	\gv{\mu}_{k}
	&=\defint{}{}{\h_k\N[\xk]{\m_{k|k-1}}{\P_{k|k-1}}}{\xk}\label{eq:update_mean_intergral}\\
	\v{S}_{k}
	&=\defint{}{}{\fparen[\big]{\h_k-\gv{\mu}_k}\fparen[\big]{\h_k-\gv{\mu}_k}^\tr\N[\xk]{\m_{k|k-1}}{\P_{k|k-1}}}{\xk}+\v{R} \label{eq:update_variance_intergral}\\
	\v{C}_{k}
	&=\defint{}{}{\fparen[\big]{\xk-\m_{k|k-1}}\fparen[\big]{\h_k-\gv{\mu}_k}^\tr\N[\xk]{\m_{k|k-1}}{\P_{k|k-1}}}{\xk} \label{eq:update_covariance_intergral}.
\end{align}
The approximation to the filtering distribution $\Pdf{\xk}{\y_{1:k}}\approx \N[\xk]{\m_{k|k}}{\P_{k|k}}$ 
is then given by applying Lemma~\ref{lemma:gaussian_cond} to \eqref{eq:joint_update_approximation}.
Then, analogously to the update equations of the Kalman filter \eqref{eq:Kalman_filter}, we get
\begin{align}
	\v{v}_k&=\yk-\gv{\mu}_k\\
	\v{K}_k&=\v{C}_{k}\v{S}_{k}^{-1}\\
	\v{m}_{k|k}&=\v{m}_{k|k-1}+\v{K}_k\v{v}_k\\
	\v{P}_{k|k}&=\v{P}_{k|k-1}-\v{K}_k\v{S}_k\v{K}_{k}^\tr
\end{align}

\subsubsection*{Smoothing step}

Let us write down the approximation to a 
conditional distribution that is easily derived from Equation~\eqref{eq:joint_predictive_approximation}, 
namely (note the change in indexing):
\begin{align}
	\Pdf{\xk}{\x_{k+1},\y_{1:T}}=\Pdf{\xk}{\x_{k+1},\y_{1:k}}&\approx \N[\xk]{\m'_{k}}{\P'_{k}}\\
	\m'_{k} &= \m_{k|k}+\v{G}_k\fparen*{\x_{k+1}-\m_{k+1|k}}\\
	\P'_{k} &= \P_{k|k}-\v{G}_k\P_{k+1|k}\v{G}_k^\tr\\
	\v{G}_k &= \v{C}_{k,k+1}\v{P}_{k+1|k}^{-1}.
	\label{eq:middle_conditional}
\end{align}
At this point we have derived all the components needed to compute
\eqref{eq:smooth_joint_bayes}. As pointed out previously, the last, i.e $T$:th, filtering 
distribution is also the ``first" smoothing distribution, and smoothing recursions
then advance backwards in time. Let us then assume that the smoothing
distribution of the previous step, $\Pdf{\x_{k+1}}{\y_{1:T}}$, is available. Then
by applying Lemma~\ref{lemma:gaussian_joint}, we have
\begin{align}
	\Pdf{\xk,\x_{k+1}}{\Y}&=\Pdf{\xk}{\x_{k+1},\y_{1:T}}\Pdf{\x_{k+1}}{\y_{1:T}}\\
	&\approx
	\N[\bm{\xk\\\x_{k+1}}]{
	\begin{bmatrix}
		\m_{k|T}\\
		\m_{k+1|T}
	\end{bmatrix}}{
	\begin{bmatrix}
		\P_{k|T}&\v{D}_{k}\\
		\v{D}_{k}^\tr&\P_{k+1|T}
	\end{bmatrix},
	}\label{eq:joint_smoothing}
\end{align}
where
\begin{align}
	\v{D}_{k}&=\v{G}_{k}\P_{k+1|T}\\
	\m_{k|T}&=\m_{k|k}+\v{G}_{k}\left(\m_{k+1|T}-\m_{k+1|k}\right)\\
	\P_{k|T}&=\P_{k|k}+\v{G}_{k}\left(\P_{k+1|T}-\P_{k+1|k}\right)\v{G}_{k}^\tr.
\end{align}%
%
What we have now established is that a Gaussian assumed density approximation to the joint 
smoothing distributions \eqref{eq:smooth_joint_bayes} is transformed into solving six
multidimensional integrals of form \emph{nonlinear function} $\times$ \emph{Gaussian}, 
namely the ones in
\eqref{eq:prediction_mean_intergral}, \eqref{eq:prediction_variance_intergral}, \eqref{eq:prediction_cov_intergral},
\eqref{eq:update_mean_intergral}, \eqref{eq:update_variance_intergral} and
\eqref{eq:update_covariance_intergral}. 
Notably, the smoothing distribution approximations can be computed
without further integrations.

\subsubsection{Numerical integration approach}
 We will now discuss the topic of numerically solving Gaussian 
 expectation integrals of the form
 \begin{align}
	\E{\F{\gv{\kappa}}{\x}}&\equiv \defint{\mathcal{X}}{}{\F{\gv{\kappa}}{\x}\N[\x]{\m}{\P}}{\x}
	\label{eq:gauss_integral}
\end{align}%
where it is assumed that 
\begin{align}
	\mathcal{X} &= \R^{d_x}\\
	\x &\sim \N{\m}{\P}\\
	\defint{}{}{\abs{\F{\gv{\kappa}}{\x}\N[\x]{\m}{\P}}}{\x} & < \infty. 
\end{align}

%Numerical integration in one dimension is known as \emph{quadrature} and in higher dimensions as \emph{cubature}.
As explained in \textcite{Wu2006}, the approaches to solving \eqref{eq:gauss_integral}
can be justifiably divided into three categories: 
\begin{enumerate}[i)] \addtolength{\leftskip}{.5cm} \itemsep1pt \parskip0pt \parsep0pt
  \item product rules
  \item rules exact for monomials
  \item integrand approximations.
\end{enumerate}
Recognizing that the chosen numerical integration method is the principal differentiator provides a 
common framework for analyzing the properties of the numerous Gaussian filters and smoothers 
\parencite{Sarkka2010, Sarkka2008a,Ito2000,Wu2006,julier1997new,Arasaratnam2009,Arasaratnam2011}.
Furthermore the first two categories differ only in their approach to multidimensional integrals,
so that main difference between the categories can be described
as applying an integration formula known to be exact for certain class of integrands
or approximating the integrand and integrating the approximation exactly.
Since truncated Taylor series approximations are often used in the latter case, an important distinction
is that the former does not require computation of Jacobians or higher order differentials. 

Since there exists many efficient integration rules defined on the unidimensional line,
a natural idea is to extend these to the hypercube by iterated integrals. This is exactly the 
basic premise of the product rules. The most efficient polynomial interpolation type of rules
in one dimension are known as \emph{Gauss' quadrature} rules and the subset
for Gaussian weighted integrals are called the \emph{Gauss-Hermite}
quadrature rules. Quadrature is a term referring to unidimensional numerical integration, whereas
\emph{cubature} is the generalization to higher dimensions. A common form for cubature rules is
\begin{align}
	\defint{}{}{&\F*{l}{\x}\N[\x]{\m}{\P}}{\x} \approx \sum_{i=1}^m w_i\F*{l}{\v{u}_i},
\end{align}
where the points of evaluation $\brac*{\v{u}_i}_{i=1}^m$ are called the \emph{sigma points}
(or \emph{abscissas} or just points) and $w_i$ are the weights. As is expected from the iterated integration approach,
the problem with product rules is the exponential increase in the number of sigma points with
the number of dimensions, also known as the \emph{curse of dimensionality}. Thus if the unidimensional rule has $m$ sigma points (and thus $m$ integrand evaluations),
then the $d$ dimensional product rule has $m^d$ sigma points. The Gaussian filter based
on Gauss-Hermite product rules is known simply as the Gauss-Hermite Kalman filter (GHKF, sometimes shortened also GKF or QKF, for quadrature Kalman filter) \parencite{Ito2000}.
The number of sigma points in the unidimensional rule is a parameter of GHKF.

More sophisticated cubature methods search for rules
exact for \emph{monomials} $\prod_{j=1}^{d_x} x_j^{e_j}$, where $\x=\brak[\big]{x_1,\dots,x_{d_x}}^\tr$. 
The \emph{degree} of the monomial is defined as $\sum_j e_j$ and a cubature rule is then said to have
\emph{precision} $p$, if it integrates exactly monomials up to degree $p$ but not
to degree $p+1$. Naturally since integration is a linear operation, a rule which is exact 
for a monomial up to order $o$ is exact for multidimensional polynomials of order $o$.
Unfortunately finding efficient rules exact for monomials is something of an art, since
even the least possible number of points required for given precision and dimension is in many
instances unknown. Nevertheless, following the work in \textcite{Wu2006}, in \textcite{Arasaratnam2009,Arasaratnam2011} a filter
and a corresponding smoother are presented, based on a third degree 
cubature rule. The theoretical lower bound in points for a third degree rule
is $2\,d_x$, which is met by the rule used in the \emph{Cubature Kalman Filter} (CKF) \todo{present the rule} and
the \emph{Cubature Kalman Smoother} (CKS). Another notable nonlinear filter in this
category is the \emph{Unscented Kalman Filter} (UKF) \parencite{julier1997new,Merwe2004}, 
also based on a third degree rule. A corresponding smoother is derived in \textcite{Sarkka2008a}.

The oldest and most well known nonlinear filter, belonging to the third category, is
the \emph{extended Kalman filter} (EKF) \parencite{jazwinski1970stochastic}. 
It is based on forming local linear approximations to the dynamic
and measurement models so that the standard linear Kalman filter equations can be used.
An undesirable requirement of the EKF is that it requires computing
the Jacobian matrices of $\f$ and $\h$.

