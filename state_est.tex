\subsection{Linear-Gaussian State Space Models}

Since linear mappings can be described by matrices, stationary linear-Gaussian SSMs 
are described by the subset of SSMs of the form \eqref{eq:ssm_general} where  
\begin{align}
	\v{f}_{\Th}\left(\x_{k-1}\right)&=\v{A}\x_{k-1}\\
	\v{h}_{\Th}\left(\xk\right)&=\v{H}\xk
\end{align}
\todo{Explain something about linear-Gaussian SSMs}

\subsubsection{Kalman filter}

The recursions are as follows \parencite{Kalman1960,jazwinski2007stochastic}:
\todo{Elaborate on the Kalman filter}
\begin{subequations}
\label{eq:Kalman_filter}
\begin{description}
\addtolength{\leftskip}{1cm}
\item[Predict:]
\begin{align}
	\v{m}_{k|k-1}&=\v{A}\v{m}_{k-1|k-1}\\
	\v{P}_{k|k-1}&=\v{A}\v{P}_{k-1|k-1}\v{A}^\tr+\v{Q}
\end{align}
\item[Update:]
\begin{align}
	\v{v}_k&=\v{y}_k-\v{H}\v{m}_{k|k-1}\\
	\v{S}_k&=\v{H}\v{P}_{k|k-1}\v{H}^\tr+\v{R}\\
	\v{K}_k&=\v{P}_{k|k-1}\v{H}^\tr\v{S}_{k}^{-1}\\
	\v{m}_{k|k}&=\v{m}_{k|k-1}+\v{K}_k\v{v}_k\\
	\v{P}_{k|k}&=\v{P}_{k|k-1}-\v{K}_k\v{S}_k\v{K}_{k}^\tr
\end{align}
\end{description}
\end{subequations}
This includes the sufficient statistics for the $T$
joint distributions 
\begin{align}
	\Pdf{\v{x}_k,\v{y}_k}{\v{y}_{1:k-1},\gv{\theta}}
	&=\N[
	\begin{bmatrix}
		\v{x}_k\\\v{y}_{k}
	\end{bmatrix}
	]{
	\begin{bmatrix}
		\v{m}_{k|k-1}\\
		\v{H}\v{m}_{k|k-1}
	\end{bmatrix}
	}{
	\begin{bmatrix}
		\v{P}_{k|k-1} & \v{P}_{k|k-1}\v{H}^\tr\\
		\v{H}\v{P}_{k|k-1}^\tr & \v{S}_k  
	\end{bmatrix}
	}
	\label{eq:joint_per_kalmanstep}
\end{align}

\subsubsection{Rauch-Tung-Striebel Smoother}

The standard Rauch-Tung-Striebel (RTS) smoother gives the statistics $\v{m}_{k|T}$ and $\v{P}_{k|T}$ \parencite{jazwinski2007stochastic,Rauch1965}.
\begin{subequations}
\begin{align}
	\v{J}_k&=\v{P}_{k|k}\v{A}^\tr\v{P}_{k|k+1}^{-1}\\
	\v{m}_{k|T}&=\v{m}_{k|k}+\v{J}_k\left(\v{m}_{k+1|T}-\v{m}_{k+1|k}\right)\\
	\v{P}_{k|T}&=\v{P}_{k|k}+\v{J}_k\left(\v{P}_{k+1|T}-\v{P}_{k+1|k}\right)\v{J}_k^\tr
	%\v{C}_{k|T}&=\v{P}_{k|k}\v{J}_{k-1}^\tr+\v{J}_k\left(\v{C}_{k+1|T}-\v{A}\v{P}_{k|k}\right)\v{J}_{k-1}^\tr \label{eq:rts_cross_timestep_covariance}
\end{align}
\end{subequations}
\begin{align}
\begin{split} 
	\Pdf{\v{x}_k, \v{x}_{k-1}}{\v{Y},\Th}&=
	\N[\bm{\v{x}_k\\\v{x}_{k-1}}]{
	\bm{
		\v{m}_{k|T}\\
		\v{m}_{k-1|T}
	}
	}{
	\bm{
		\v{P}_{k|T} & \v{P}_{k|T}\v{J}_k^\tr\\
		\v{J}_k\v{P}_{k|T} & \v{P}_{k-1|T}  
	}
	}
\end{split}
\label{eq:sum_expectations}
\end{align}

In \parencite{Elliott1999} a new kind of filter is presented that
can compute \eqref{eq:sum_expectations} with only forward recursions. 
\todo{Elaborate on the RTS smoother}
\parencite{Paninski2010}

%%%%%%%%%%%%%%%%%%%%%%%%%%%%%%%%%%%%%%%%%
\subsection{Nonlinear-Gaussian SSMs}%%%%%
%%%%%%%%%%%%%%%%%%%%%%%%%%%%%%%%%%%%%%%%%
\label{sec:nonlinear_state}
In the nonlinear case at least one of the mappings $\v{f}_{\Th}$ and $\v{h}_{\Th}$ in
\eqref{eq:ssm_general} is nonlinear. Unfortunately in this case computing the filtering
distributions in closed form becomes intractable and one has to resort to 
some sort of approximations. Following \textcite{Arasaratnam2009}, we can
divide these approximate filtering (and smoothing) solutions into two 
categories: 
\begin{enumerate}[i)] \addtolength{\leftskip}{.5cm} \itemsep1pt \parskip0pt \parsep0pt
  \item \emph{Local approaches} assume the parametric form of the posterior
  distributions \eqref{eq:pred_bayes}, \eqref{eq:filt_bayes} and \eqref{eq:smooth_bayes} \emph{a priori}. 
  These  methods are analytically inexact but less computationally demanding. This is the category that
we will be concerned with in this thesis. 
  \item \emph{Global approaches} require the use of particle filtering (or sequential Monte Carlo), 
  which is asymptotically exact but computationally demanding
\end{enumerate}%
%
The number of different methods in the first category is substantial,
but a large proportion can be analyzed under the framework of
\emph{Gaussian filtering} (or assumed density filtering
with a Gaussian assumption). As will be shown later, the specific Gaussian filtering 
methods only differ in their chosen numerical integration methods.  

\subsubsection{Gaussian filtering and smoothing}

One approach to forming Gaussian approximations is to assume a Gaussian
probability distribution a priori \parencite{Ito2000,Wu2006,Sarkka2010}. 
Since a Gaussian distribution is 
defined by its first two moments, a moment matched approximation
can be obtained if the first two moments of the actual probability
distribution can be computed \parencite{Ito2000,Sarkka2006}. As will
be seen, computing these approximations reduces to the problem
of computing multidimensional moment integrals of the form 
\emph{nonlinear function} $\times$ \emph{Gaussian}. We will next derive 
the general form of the three moment integrals and then show how they can be applied
in the specific case of approximating the joint smoothing distribution
of equation \eqref{eq:smooth_joint_bayes}.

Analogously to equation~\eqref{eq:ssm_distr_general}, let 
\begin{align}
	\Pdf{\v{a}}&= \N{\m}{\gv{\Sigma}_a} \label{eq:pa}\\
	\Pdf{\v{b}}{\v{a}}&= \N{\F*{f}{\v{a}}}{\gv{\Sigma}_{b|a}} \label{eq:pb_given_a}
\end{align}
then 
\begin{align}
	\label{eq:joint_nongaussian}
	\Pdf{\v{a},\v{b}}&=\Pdf{\v{b}}{\v{a}}\Pdf{\v{a}}=\N[\v{b}]{\F*{f}{\v{a}}}{\gv{\Sigma}_{b|a}}
	\N[\v{a}]{\v{m}}{\gv{\Sigma}_a}
\end{align}
is only Gaussian if $\F*{f}{\v{a}}$ is linear. Assuming that's not the case,
let us denote a Gaussian approximation
to \eqref{eq:joint_nongaussian} with
\begin{align}
	\label{eq:joint_gaussian_approx}
	\Pdf{\begin{bmatrix}
		\v{a}\\
		\v{b}
	\end{bmatrix}}&\approx
	\N{
	\begin{bmatrix}
		\gv{\mu}_a\\
		\gv{\mu}_b
	\end{bmatrix}
	}{
	\begin{bmatrix}
		\gv{\Sigma}_{aa}&\gv{\Sigma}_{ab}\\
		\gv{\Sigma}_{ab}^\tr&\gv{\Sigma}_{bb}
	\end{bmatrix}
	}
\end{align}
Then for a moment matched approximation we have to have
\begin{align}
	\gv{\mu}_a&=\v{m}&
	\gv{\Sigma}_{aa}&=\gv{\Sigma}_{a}\\
	\gv{\mu}_b&=\defint{}{}{\v{b}\,\Pdf{\v{b}}}{\v{b}} &
	\gv{\Sigma}_{bb}&=\defint{}{}{(\v{b}-\gv{\mu}_b)(\v{b}-\gv{\mu}_b)^\tr\Pdf{\v{b}}}{\v{b}} 
\end{align}
It is straightforward to show that $\gv{\mu}_b$, $\gv{\Sigma}_{bb}$ and $\gv{\Sigma}_{ab}$ 
can all be written in terms of \eqref{eq:pa} and \eqref{eq:pb_given_a}.
To see this, let us rewrite $\gv{\mu}_b$ as
\begin{align}
	\gv{\mu}_b&=\defint{}{}{\v{b}\,\Pdf{\v{b}}}{\v{b}} \nonumber\\
	&=\defint{}{}{\v{b}\Big(\!\defint{}{}{\Pdf{\v{b}}{\v{a}}\Pdf{\v{a}}}{\v{a}}\Big)}{\v{b}}\nonumber\\
	&\mbox{{\small (change the order of integration according to Fubini's theorem)}}\nonumber\\
	&=\defint{}{}{\v{f}(\v{a})\Pdf{\v{a}}}{\v{a}}\label{eq:mean_int},
\end{align}
$\gv{\Sigma}_{bb}$ as
\begin{align}
	\gv{\Sigma}_{bb}&=\defint{}{}{\v{b}\v{b}^\tr\Pdf{\v{b}}}{\v{b}}-\gv{\mu}_b\gv{\mu}_b^\tr \nonumber\\
	%&=\defint{}{}{\defint{}{}{\lbrack(\v{b}-\v{f}(\v{a}))(\v{b}-\v{f}(\v{a}))^\tr+\v{f}(\v{a})\v{f}(\v{a})^\tr\rbrack\Pdf{\v{b}}{\v{a}}}{\v{b}}\Pdf{\v{a}}}{\v{a}}-\gv{\mu}_b\gv{\mu}_b^\tr\\
	&=\defint{}{}{\v{f}(\v{a})\v{f}(\v{a})^\tr\Pdf{\v{a}}}{\v{a}}-\gv{\mu}_b\gv{\mu}_b^\tr
+\defint{}{}{\fparen*{\v{b}-\F*{f}{\v{a}}}\fparen*{\v{b}-\F*{f}{\v{a}}}^\tr\Pdf{\v{b}}{\v{a}}}{\v{a}}{\v{b}}\nonumber\\
	&=\defint{}{}{\fparen*{\F*{f}{\v{a}}-\gv{\mu}_b}\fparen*{\F*{f}{\v{a}}-\gv{\mu}_b}^\tr\Pdf{\v{a}}}{\v{a}}+\gv{\Sigma}_{b|a}\label{eq:var_int}.
\end{align}
and the cross-covariance $\gv{\Sigma}_{ab}$ as
\begin{align}
	\gv{\Sigma}_{ab}&=\defint{}{}{\fparen[\big]{\v{a}-\gv{\mu}_a}\fparen[\big]{\v{b}-\gv{\mu}_b}^\tr\Pdf{\v{a}}\Pdf{\v{b}}{\v{a}}}{\v{a}}{\v{b}} \nonumber\\
	&=\defint{}{}{\fparen[\big]{\v{a}-\gv{\mu}_a}\fparen[\big]{\defint{}{}{\v{b}\,\Pdf{\v{b}}{\v{a}}}{\v{b}}-\gv{\mu}_b}^\tr\Pdf{\v{a}}}{\v{a}} \nonumber\\
	&=\defint{}{}{\fparen[\big]{\v{a}-\gv{\mu}_a}\fparen*{\F*{f}{\v{a}}-\gv{\mu}_b}^\tr\N{\v{m}}{\gv{\Sigma}_a}}{\v{a}}\label{eq:cross_cov_int}
\end{align}%
%

\subsubsection*{Prediction step} 
Since the Gaussian approximation to
\eqref{eq:smooth_joint_bayes} will be calculated by forward
(filtering) and backward (smoothing) recursions, let us assume that we already
have available the filtering distribution of the previous step
\begin{align}
	\Pdf{\xkk}{\y_{1:k-1}} &\approx \N[\xkk]{\m_{k-1|k-1}}{\P_{k-1|k-1}}.
\end{align}
Then
\begin{align}
	\Pdf{\xkk,\xk}{\y_{1:k-1}}&\approx\N[\xkk]{\m_{k-1|k-1}}{\P_{k-1|k-1}}\N[\xk]{\f_{k-1}}{\v{Q}}\\
	&\approx
	\N[\bm{\xkk \\ \xk}]{
	\begin{bmatrix}
		\v{m}_{k-1|k-1}\\
		\v{m}_{k|k-1}
	\end{bmatrix}}{
	\begin{bmatrix}
		\v{P}_{k-1|k-1}&\v{C}_{k-1,k}\\
		\v{C}_{k-1,k}^\tr&\v{P}_{k|k-1}
	\end{bmatrix}
	}
	\label{eq:joint_predictive_approximation}
\end{align}
where by application of equations \eqref{eq:mean_int}, \eqref{eq:var_int} and \eqref{eq:cross_cov_int} 
\begin{align}
	\begin{split}
	\v{m}_{k|k-1}&=\defint{}{}{\f_{k-1}\N[\xkk]{\m_{k-1|k-1}}{\P_{k-1|k-1}}}{\xkk}\label{eq:prediction_mean_intergral}
	\end{split}\\
	\begin{split}
	\v{P}_{k|k-1}&=\defint{}{}{\fparen*{\f_{k-1}-\v{m}_{k|k-1}}\fparen*{\f_{k-1}-\m_{k|k-1}}^\tr\\
	&\qquad\times\N[\xkk]{\m_{k-1|k-1}}{\P_{k-1|k-1}}}{\xkk}+\v{Q}\label{eq:prediction_variance_intergral}
	\end{split}\\
	\begin{split}
		\v{C}_{k-1,k}&=\defint{}{}{\fparen*{\xkk-\m_{k-1|k-1}}\fparen*{\f_{k-1}-\v{m}_{k|k-1}}^\tr\\
		&\qquad\times\N{\xkk}{\m_{k-1|T}}{\P_{k-1|T}}}{\xkk}\label{eq:prediction_cov_intergral}
	\end{split}
\end{align}%
%

\subsubsection*{Update step}

For the update step we first approximate
\begin{align}
\begin{split}
	\Pdf{\xk,\yk}{\y_{1:k-1}}=\Pdf{\yk}{\xk}\Pdf{\xk}{\y_{1:k-1}}&\approx\N[\yk]{\h_k}{\v{R}}\N[\xk]{\m_{k|k-1}}{\P_{k|k-1}}\\
	&\approx 
	\N{
	\begin{bmatrix}
		\m_{k|k-1}\\
		\gv{\mu}_{k}
	\end{bmatrix}}{
	\begin{bmatrix}
		\v{P}_{k|k-1}&\v{C}_{k}\\
		\v{C}_{k}^\tr&\v{S}_{k}
	\end{bmatrix}
	}.
	\label{eq:joint_update_approximation}
\end{split}
\end{align}
Applying equations \eqref{eq:mean_int}, \eqref{eq:var_int} and \eqref{eq:cross_cov_int} again,
we get
\begin{align}
	\gv{\mu}_{k}
	&=\defint{}{}{\h_k\N[\xk]{\m_{k|k-1}}{\P_{k|k-1}}}{\xk}\label{eq:update_mean_intergral}\\
	\v{S}_{k}
	&=\defint{}{}{\fparen[\big]{\h_k-\gv{\mu}_k}\fparen[\big]{\h_k-\gv{\mu}_k}^\tr\N[\xk]{\m_{k|k-1}}{\P_{k|k-1}}}{\xk}+\v{R} \label{eq:update_variance_intergral}\\
	\v{C}_{k}
	&=\defint{}{}{\fparen[\big]{\xk-\m_{k|k-1}}\fparen[\big]{\h_k-\gv{\mu}_k}^\tr\N[\xk]{\m_{k|k-1}}{\P_{k|k-1}}}{\xk} \label{eq:update_covariance_intergral}.
\end{align}
The approximation to the filtering distribution $\Pdf{\xk}{\y_{1:k}}\approx \N[\xk]{\m_{k|k}}{\P_{k|k}}$ 
is then given by applying the well known formula for calculating the conditional distribution of jointly 
Gaussian variables, giving
\begin{align}
	\m_{k|k}
	&=\m_{k|k-1}+\v{C}_{k}\v{S}_{k}^{-1}\left(\yk-\gv{\mu}_k\right)\label{eq:update_mean}\\
	\P_{k|k}
	&=\P_{k|k-1}-\v{C}_{k}\v{S}_{k}^{-1}\v{C}_{k}^\tr. \label{eq:update_variance}
\end{align}

\subsubsection*{Smoothing step}

Let us write down the approximation to a 
conditional distribution that is easily derived from equation~\eqref{eq:joint_predictive_approximation}, 
namely (note the change in indexing):
\begin{align}
	\Pdf{\xk}{\x_{k+1},\y_{1:T}}=\Pdf{\xk}{\x_{k+1},\y_{1:k}}&\approx \N[\xk]{\m'_{k}}{\P'_{k}}\\
	\m'_{k} &= \m_{k|k}+\v{G}_k\fparen*{\x_{k+1}-\m_{k+1|k}}\\
	\P'_{k} &= \P_{k|k}-\v{G}_k\P_{k+1|k}\v{G}_k^\tr\\
	\v{G}_k &= \v{C}_{k,k+1}\v{P}_{k+1|k}^{-1}
	\label{eq:middle_conditional}
\end{align}
At this point we have derived all the components needed to compute
\eqref{eq:smooth_joint_bayes}. As pointed out previously, the last, i.e $T$:th, filtering 
distribution is also the ``first" smoothing distribution, and smoothing recursions
then advance backwards in time. Let us then assume that the smoothing
distribution of the previous step, $\Pdf{\x_{k+1}}{\y_{1:T}}$, is available. Then
\begin{align}
\begin{split}
	\Pdf{\x,\x_{k+1}}{\y_{1:T}}&=\Pdf{\xk}{\x_{k+1},\y_{1:T}}\Pdf{\x_{k+1}}{\y_{1:T}}\\
	&\approx
	\N{
	\begin{bmatrix}
		\m_{k|T}\\
		\m_{k+1|T}
	\end{bmatrix}}{
	\begin{bmatrix}
		\P_{k|T}&\v{D}_{k}\\
		\v{D}_{k}^\tr&\P_{k+1|T}
	\end{bmatrix}
	}
\end{split}
\end{align}
where
\begin{align}
	\v{D}_{k}&=\v{G}_{k}\P_{k+1|T}\\
	\m_{k|T}&=\m_{k|k}+\v{G}_{k}\left(\m_{k+1|T}-\m_{k+1|k}\right)\\
	\P_{k|T}&=\P_{k|k}+\v{G}_{k}\left(\P_{k+1|T}-\P_{k+1|k}\right)\v{G}_{k}^\tr.
\end{align}%
%
What we have now established is that a Gaussian assumed density approximation to the joint 
smoothing distributions \eqref{eq:smooth_joint_bayes} is transformed into solving six
multidimensional integrals of form \emph{nonlinear function} $\times$ \emph{Gaussian}, 
namely the ones in
\eqref{eq:prediction_mean_intergral}, \eqref{eq:prediction_variance_intergral}, \eqref{eq:prediction_cov_intergral},
\eqref{eq:update_mean_intergral}, \eqref{eq:update_variance_intergral} and
\eqref{eq:update_covariance_intergral}. 
Notably, the smoothing distribution approximations can be computed
without further integrations.

\subsubsection{Numerical integration approach}
 We will now discuss the topic of numerically solving integrals of the form
 \begin{align}
	\defint{}{}{&\F*{l}{\x}\N[\x]{\m}{\P}}{\x} \nonumber\\
	\equiv\defint{}{}{&\F*{l}{\x}\fparen*{\fparen*{2\pi}^{d_x}\det\P}^{-\sfrac{1}{2}}\exp\brak[\big]{-\frac{1}{2}(\x-\m)^\tr\P^{-1}(\x-\m)}}{\x}.
	\label{eq:gauss_integral}
\end{align}%
%
Even though omitted in the notation, it is assumed that the integral\todo{existence of the integral}
is over the unbounded space $\R^{d_x}$ of $\x$. Furthermore \eqref{eq:gauss_integral}
should be thought of as a Gaussian \emph{weighted} multidimensional integral, so that $\N[\x]{\m}{\P}$
is not considered to be part of the integrand (more generally, \eqref{eq:gauss_integral}
should be stated as the \emph{Lebesgue} integral $\defint{\R^{d_x}}{}{\F*{l}{\x}}{F}$, where $F$
defines the probability measure of the probability space).
%Numerical integration in one dimension is known as \emph{quadrature} and in higher dimensions as \emph{cubature}.
As explained in \textcite{Wu2006}, the approaches to solving \eqref{eq:gauss_integral}
can be justifiably divided into three categories: 
\begin{enumerate}[i)] \addtolength{\leftskip}{.5cm} \itemsep1pt \parskip0pt \parsep0pt
  \item product rules
  \item rules exact for monomials
  \item integrand approximations.
\end{enumerate}
Recognizing that the chosen numerical integration method is the principal differentiator provides a 
common framework for analyzing the properties of the numerous Gaussian filters and smoothers \parencite{Sarkka2010, Sarkka2008a}.
Furthermore the first two categories differ only in their approach to multidimensional integrals,
so that main difference between the categories can be described
as applying an integration formula known to be exact for certain class of integrands
or approximating the integrand and integrating the approximation exactly.
Since truncated Taylor series approximations are often used in the latter case, an important distinction
is that the former doesn't require computation of Jacobians or higher order differentials. 

Since there exists many efficient integration rules defined on the unidimensional line,
a natural idea is to extend these to the hypercube by iterated integrals. This is exactly the 
basic premise of the product rules. The most efficient polynomial interpolation type of rules
in one dimension are known as \emph{Gauss' quadrature} rules and the subset
for Gaussian weighted integrals are called the \emph{Gauss-Hermite}
quadrature rules. Quadrature is a term referring to unidimensional numerical integration, whereas
\emph{cubature} is the generalization to higher dimensions. A common form for cubature rules is
\begin{align}
	\defint{}{}{&\F*{l}{\x}\N[\x]{\m}{\P}}{\x} \approx \sum_{i=1}^m w_i\F*{l}{\v{u}_i}, 
\end{align}
where the points of evaluation $\brac*{\v{u}_i}_{i=1}^m$ are called the \emph{sigma points}
(or just points) and $w_i$ are the weights. As is expected from the iterated integration approach,
the problem with product rules is the exponential increase in the number of sigma points with
the number of dimensions, also known as the \emph{curse of dimensionality}. Thus if the unidimensional rule has $m$ sigma points (and thus $m$ integrand evaluations),
then the $d$ dimensional product rule has $m^d$ sigma points. The Gaussian filter based
on Gauss-Hermite product rules is know simply as the Gauss-Hermite Kalman filter (GHKF) \parencite{Ito2000}.
The number of sigma points in the unidimensional rule is a parameter of GHKF.

More sophisticated cubature methods search for rules
exact for \emph{monomials} $\prod_{j=1}^{d_x} x_j^{e_j}$, where $\x=\brak[\big]{x_1,\dots,x_{d_x}}^\tr$. 
The \emph{degree} of the monomial is defined as $\sum_j e_j$ and a cubature rule is then said to have
\emph{precision} $p$, if it integrates exactly monomials up to degree $p$ but not
to degree $p+1$. Naturally since integration is a linear operation, a rule which is exact 
for a monomial up to order $o$ is exact for multidimensional polynomials of order $o$.
Unfortunately finding efficient rules exact for monomials is something of an art, since
even the least possible number of points required for given precision and dimension is in many
instances unknown. Nevertheless in \textcite{Arasaratnam2009,Arasaratnam2011} a filter
and a corresponding smoother are presented, which are based on a third degree 
cubature rule. The theoretical lower bound in points for a third degree rule
is $2\,d_x$, which is met by the rule used in the \emph{Cubature Kalman Filter} (CKF) and
the \emph{Cubature Kalman Smoother} (CKS). Another notable nonlinear filter in this
category is the \emph{Unscented Kalman Filter} (UKF) \textcite{julier1997new,Merwe2004}, 
also based on a third degree rule. A corresponding smoother is derived in \textcite{Sarkka2008a}.

The oldest and most well known nonlinear filter, belonging to the third category, is
the \emph{extended Kalman filter} (EKF) \parencite{jazwinski2007stochastic}. 
It is based on forming local linear approximations to the dynamic
and measurement models so that the standard linear Kalman filter equations can be used.
An undesirable requirement of the EKF is that it requires computing
the Jacobian matrices of $\f$ and $\h$.

