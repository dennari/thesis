%% Korjaa vastaamaan korkeakouluasi
%%
%% Change the school field to describe your school 
\university{aalto university}{aalto-yliopisto}
\school{School of Electrical Engineering}{Sähkötekniikan korkeakoulu}

%% Vain kandity�lle: Korjaa seuraavat vastaamaan tutkinto-ohjelmaasi
%%
%% Only for B.Sc. thesis: Choose your degree programme. 
\degreeprogram{Electronics and electrical engineering}%
{Elektroniikka ja sähkötekniikka}
%%

%% Vain DI/M.Sc.- ja lisensiaatinty�lle: valitse laitos, 
%% professuuri ja sen professuurikoodi. 
%%
%% Only for M.Sc. and Licentiate thesis: Choose your department,
%% professorship and professorship code. 
\department{Department of Biomedical Engineering and Computational Science}%
{Lääketieteellisen tekniikan ja laskennalisen tieteen laitos}
\professorship{Computational and Cognitive Biosciences}{Laskennallinen ja
kognitiivinen biotiede}
\code{S-114}
%%

%% Valitse yksi n�ist� kolmesta
%%
%% Choose one of these:
\univdegree{MSc}

%% Oma nimi
%%
%% Should be self explanatory...
\author{Ville Väänänen}

%% Opinn�ytteen otsikko tulee vain t�h�n. �l� tavuta otsikkoa ja
%% v�lt� liian pitk�� otsikkoteksti�. Jos latex ryhmittelee otsikon
%% huonosti, voit joutua pakottamaan rivinvaihdon \\ kontrollimerkill�.
%% Muista ett� otsikkoja ei tavuteta! 
%% Jos otsikossa on ja-sana, se ei j�� rivin viimeiseksi sanaksi 
%% vaan aloittaa uuden rivin.
%% 
%% Your thesis title. If the title is very long and the latex 
%% does unsatisfactory job of breaking the lines, you will have to
%% break the lines yourself with \\ control character. 
%% Do not hyphenate titles.
\thesistitle{Gaussian filtering and smoothing based parameter estimation in nonlinear models for sequential data}
{Gaussiseen suodatukseen ja siloitukseen perustuva parametrien estimointi epälineaarisissa aikasarjamalleissa}

\place{Espoo}
%% Kandidaatinty�n p�iv�m��r� on sen esitysp�iv�m��r�! 
%% 
%% For B.Sc. thesis use the date when you present your thesis. 
\date{\today}

%% Kandidaattiseminaarin vastuuopettaja tai diplomity�n valvoja.
%% Huomaa titteliss� "\" -merkki pisteen j�lkeen, 
%% ennen v�lily�nti� ja seuraavaa merkkijonoa. 
%% N�in tehd��n, koska kyseess� ei ole lauseen loppu, jonka j�lkeen tulee 
%% hieman pidempi v�li vaan halutaan tavallinen v�li.
%%
%% B.Sc. or M.Sc. thesis supervisor 
%% Note the "\" after the comma. This forces the following space to be 
%% a normal interword space, not the space that starts a new sentence. 
\supervisor{Prof.\ Jouko Lampinen}{Prof.\ Jouko Lampinen}

%% Kandidaatinty�n ohjaaja(t) tai diplomity�n ohjaaja(t)
%% 
%% B.Sc. or M.Sc. thesis instructor(s)
%\instructor{Prof. Pirjo Professori}{Prof. Pirjo Professori}
\instructor{D.Sc.\ (Tech.) Simo Särkkä}{TkT Simo Särkkä}
%\instructor{M.Sc.\ (Tech.) Polli Pohjaaja}{DI Polli Pohjaaja}

%% Aaltologo: syntaksi: \uselogo{red|blue|yellow}{?|!|''}
%% Logon kieli on sama kuin dokumentin kieli
%%
%% Aalto logo: syntax: \uselogo{red|blue|yellow}{?|!|''} 
%% Logo language is set to be the same as the document language.
\uselogo{aaltoRed}{''}

%% Tehd��n kansilehti
%%
%% Create the coverpage
