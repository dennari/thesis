
%% English abstract, uncomment if you need one. 
%% 
%% Abstract keywords
\keywords{Parameter estimation, Sequential data, Nonlinear state space models,
Expectation maximization, Quasi--Newton optimization}
%% Abstract text
\begin{abstractpage}[english]
State space modeling is a widely used statistical approach
for sequential data. Commonly the resulting models can be considered to contain
two interconnected estimation problems: that of the dynamic states
and that of the static parameters. The difficulty of these problems
depends critically on the linearity of the model, either with
respect to the states, the parameters or both.\\\\%
%
In this thesis we show how to obtain maximum likelihood and maximum a posteriori
estimates for the static parameters. Two methods are considered: gradient based nonlinear
optimization of the marginal log-likelihood and expectation maximization.
The former requires the filtering distributions and the latter both the
filtering and the smoothing distributions.
We show how the efficient Gaussian filtering based methods
can be applied to obtain these distributions when the model
is nonlinear.\\\\%
%
The resulting optimization equations are demonstrated in a linear model
with simulated data and a nonlinear model with actual photoplethysmograph
data. 

\end{abstractpage}

\newpage
%% Note that 
%% if you are writting your master's thesis in English place the English
%% abstract first followed by the possible Finnish abstract

%% Suomenkielinen tiivistelmä
%% 
%% Finnish abstract
%%
%% Tiivistelmän avainsanat
\keywords{Parametrien estimointi, Aikasarjat, Epälineaariset tila-avaruusmallit,
EM, Kvasi--Newton optimointi}
%% Tiivistelmän tekstiosa
\begin{abstractpage}[finnish]
Tila-avaruusmallinnus on eräs laajalti käytetty aikasarjojen mallinnusmenetelmä.
Tila-avaruusmallin voidaan ajatella sisältävän kaksi keskenään
vuorovaikkuteista estimointiongelmaa: dynaamisten tilojen estimointi
sekä staattisten parametrien estimointi. Näiden estimointiongelmien
vaikeuteen vaikuttaa erityisen paljon mallin lineaarisuus -- sekä
tilojen että parametrien suhteen.\\\\%
%
Tässä diplomityössä näytämme, kuinka 
suurimman uskottavuuden estimaattori, jonka voidaan ajatella olevan
erikoistapaus a posteriori tiheysfunktion maksimoivasta estimaattorista, 
voidaan johtaa tila-avaruusmallin staattisille parametreille.
Vertailemme kahta eri menetelmää: 
uskottavuusfunktion gradienttipohjaista epälineerista optimointia
sekä expectation maximization algoritmiä.\\\\%
%
Lopputuloksina saatuja optimointialgoritmeja sovelletaan kahdessa
eri tapauksessa, joista toisessa käytetään lineaarista
mallia ja simuloitua dataa ja toisessa epälineaarista mallia
ja oikeaa mittalaitteesta peräisin olevaa dataa.


\end{abstractpage}

%% Pakotetaan uusi sivu varmuuden vuoksi, jotta 
%% mahdollinen suomenkielinen ja englanninkielinen tiivistelmä
%% eivät tule vahingossakaan samalle sivulle
%%
%% Force new page so that English abstract starts from a new page
%



