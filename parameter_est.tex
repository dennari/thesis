%%%%%%%%%%%%%%%%%%%%%%%%%%%%%%%%%%%%%%%%%%%%%%%%%%%%%%%%%%%%%%%%%%%%%%%%%
\subsection{Maximum likelihood and maximum a posteriori estimation}%%%%%%
%%%%%%%%%%%%%%%%%%%%%%%%%%%%%%%%%%%%%%%%%%%%%%%%%%%%%%%%%%%%%%%%%%%%%%%%%

In the Bayesian sense the complete answer to the parameter estimation
problem is the marginal posterior probability of the parameters
given the measurements, which is given by Bayes' rule as
\begin{align}
	\Pdf{\gv{\theta}}{\Y}&=\frac{\Pdf{\Y}{\gv{\theta}}\Pdf{\gv{\theta}}}{\Pdf{\Y}}.
	\label{eq:param_post}
\end{align}

Computing the posterior distribution of the parameters is usually intractable \todo{why? examples?}. A much
easier problem is finding a suitable \emph{point estimate} $\hat{\gv{\theta}}$.
This effectively means that we don't need to worry about the normalizing
term $\Pdf{\Y}$, since it's constant with respect to the parameters. 
A point estimate that maximizes the posterior distribution
is called a \emph{maximum a posteriori} (MAP) estimate. 
Since the logarithm is a strictly monotonic function, maximizing a function
is the same as maximizing its logarithm. Thus the MAP estimate $\Th^*$ is given by 
\begin{align}
	\Th_{\text{MAP}} &= \text{argmax}_{\Th}\left[\underbrace{\log \Pdf{\Y}{\gv{\theta}}}_{\Pdf[\ell][2pt]{\Th}} + \log\Pdf{\gv{\theta}}\right]
	\label{eq:MAP}
\end{align}

In the case of a flat (constant and thus improper)
prior distribution, $\Pdf{\Th}$, the MAP estimate converges to the
\emph{maximum likelihood} (ML) estimate
\begin{align}
	\Th_{\text{ML}} &= \text{argmax}_{\Th}\left[\lLH\right]
	\label{eq:ML}
\end{align}
Going further we we will only be concerned with finding the ML estimate, but it should
be remembered that both of the methods we consider can be extended
to the estimation of the MAP estimate in a straightforward fashion.
\todo{Explain MAP in both cases}

Among different point estimates, the maximum likelihood estimator has good statistical properties.
Let us denote the true parameter value, the value that the data was generated with, with $\Th_\star$ and 
let $T$ denote the amount of observations.
Then provided that some conditions of not very restricting nature hold, we can state the following asymptotic properties 
for the ML estimate $\Th_{\text{ML}}$:\todo{Modify to reflect p.465 in Cappé}
\begin{description}
\addtolength{\leftskip}{1cm}
\item[Strong consistency]\hfill\\
An important property for an estimator, which says that
the estimator tends to the true value as the amount of data tends to infinity:
\begin{align}
	%\forall \Th \in \Theta\quad \frac{1}{n}\Pdf[\ell_n]{\Th} \xrightarrow{\mathrm{a.s.}} \lLH, \mathrm{when} n\to\infty
	\ell_T\left(\Th_{\text{ML}}\right) \xrightarrow{\mathrm{a.s.}} \ell\left(\Th_\star\right),\quad \mathrm{when}\;T\to\infty,
\end{align}
where $\ell_T$ is the likelihood function after $T$ measurements and $\ell$ is a continuous
deterministic function with a unique global maximum at $\Th_\star$.
%where $\Pdf[\ell_n]{\Th}$ is the log-likelihood given $n$ observations and $\lLH$ is a continuous deterministic
%function with a unique global maximum at $\Th_\star$.
\item[Asymptotic normality]\hfill\\
This property gives us the means to compute asymptotic error bounds for
the estimate:
\begin{align}
	\sqrt{T}\left(\Th_{\text{ML}}-\Th_\star\right) \xrightarrow{D} \N{\v{0},\mathcal{I}^{-1}\left(\Th_\star\right)},\quad \mathrm{when}\;T\to\infty,
	\label{tablelabel}
\end{align}
where $\mathcal{I}\left(\Th_\star\right)$ is the \emph{Fischer information matrix} evaluated at $\Th_\star$ 
\item[Efficiency]\hfill\\
When the amount of information tends to infinity, the ML-estimate achieves
the Cramér-Rao lower bound, i.e. no other consistent estimator has lower asymptotic mean-squared-error.
\end{description}


\subsubsection{Identifiability}

Intuitively, any parameters $\Th,\:\Th' \in \Theta$ cannot be distinguished
from each other with maximum likelihood estimation if
\begin{align}
	\Pdf{\Y}{\Th}&=\Pdf{\Y}{\Th'},
\end{align}
i.e., if the same data can arise with two (or more) separate
parameter values.
\todo{Elaborate on identifiability}
\parencite{Haykin2001,Cappe2005}

%%%%%%%%%%%%%%%%%%%%%%%%%%%%%%%%%%%%%%%%%%%%%%%%%%%%%%%%%%%%%%%%%%%%%
\subsection{Gradient based nonlinear optimization}\label{sec:grad}%%%
%%%%%%%%%%%%%%%%%%%%%%%%%%%%%%%%%%%%%%%%%%%%%%%%%%%%%%%%%%%%%%%%%%%%%

This is the classical way of solving the parameter estimation problem. It consists
of computing the gradient of the log-likelihood function $\lLH$ and then using some
non-linear optimization method to find a \emph{local} maximum to it 
\parencite{Mbalawataa,Cappe2005}. 
An efficient non-linear optimization algorithm is the scaled 
conjugate gradient method \parencite{Mbalawataa}.

By marginalizing the joint distribution of equation~\eqref{eq:joint_per_kalmanstep}
we get 
\begin{align}
	\Pdf{\y_k}{\y_{1:k-1},\Th}&=\N[\yk]{\v{H}\m_{k|k-1},\v{S}_k }.
\end{align}
Applying equation~\eqref{eq:lh_factorization} and taking the logarithm then gives
\begin{align}
	\lLH&=-\frac{1}{2}\sum_{k=1}^T\log\abbs{\v{S}_k}
	-\frac{1}{2}\sum_{k=1}^T\left(\v{y}_k-\v{H}\v{m}_{k|k-1}\right)^\tr\v{S}_{k}^{-1}\left(\v{y}_k-\v{H}\v{m}_{k|k-1}\right)+C,
	\label{eq:logLH}
\end{align}
where $C$ is a constant that doesn't depend on $\gv{\theta}$ and thus can
be ignored in the maximization.
Employing an efficient numerical optimization method generally
requires that the gradient of the objective function is available. 
There are at least two seemingly quite different methods for computing
the gradient of $\lLH$. The first one proceeds straightforwardly by taking the
partial derivatives of $\lLH$. As will soon be demonstrated, this leads
to some additional recursive formulas which allow computing
the gradient in parallel with the Kalman filter. The second method needs
the smoothing distributions with the cross-timestep covariances of equation~\eqref{eq:rts_cross_timestep_covariance}
and it can be easily computed with the expectation maximization machinery
that will be introduced later. These two methods can be proved to compute
the exact same quantity. At this point we will focus on the first one. 

In order to calculate the gradient of $\lLH$, we can take the partial
derivatives of it w.r.t every parameter $\theta_i$ in $\gv{\theta}$:

\begin{align}
\begin{split}
	\dpd{\lLH}{\theta_i}
	=&-\frac{1}{2}\sum_{k=1}^T\mathrm{Tr}\left(\v{S}_{k}^{-1}\dpd{\v{S}_k}{\theta_i}\right)\\
	&+\sum_{k=1}^T\left(\v{H}_k\dpd{\v{m}_{k|k-1}}{\theta_i}\right)^\tr\v{S}_{k}^{-1}\left(\v{y}_k-\v{H}\v{m}_{k|k-1}\right)\\
	&+\frac{1}{2}\sum_{k=1}^T\left(\v{y}_k-\v{H}\v{m}_{k|k-1}\right)^\tr\v{S}_{k}^{-1}\left(\dpd{\v{S}_k}{\theta_i}\right)\v{S}_{k}^{-1}\left(\v{y}_k-\v{H}\v{m}_{k|k-1}\right)\\
	\label{eq:dlogLH}
\end{split}
\end{align}
From the Kalman filter recursions \eqref{eq:Kalman_filter} we find out that 
\begin{align}
	\dpd{\v{S}_k}{\theta_i}&=\v{H}\dpd{\v{P}_{k|k-1}}{\theta_i}\v{H}+\dpd{\v{R}}{\theta_i}
\end{align}
so that we're left with the task of determining the partial derivatives for
$\v{m}_{k|k-1}$ and $\v{P}_{k|k-1}$:

\begin{align}
	\dpd{\v{m}_{k|k-1}}{\theta_i}&=\dpd{\v{A}}{\theta_i}\v{m}_{k-1|k-1}+\v{A}\dpd{\v{m}_{k-1|k-1}}{\theta_i} \label{eq:m_pred_pd}\\
	\begin{split}
	\dpd{\v{P}_{k|k-1}}{\theta_i}&=\dpd{\v{A}}{\theta_i}\v{P}_{k-1|k-1}\v{A}^\tr+\v{A}\dpd{\v{P}_{k-1|k-1}}{\theta_i}\v{A}^\tr\\
	&+\v{A}\v{P}_{k-1|k-1}\left(\dpd{\v{A}}{\theta_i}\right)^\tr+\dpd{\v{Q}}{\theta_i} \label{eq:P_pred_pd}
	\end{split}
\end{align}
as well as for $\v{m}_{k|k}$ and $\v{P}_{k|k}$:
\begin{align}
	\dpd{\v{K}_k}{\theta_i}&=\dpd{\v{P}_{k|k-1}}{\theta_i}\v{H}^\tr\v{S}_{k}^{-1}-\v{P}_{k|k-1}\v{H}^\tr\v{S}_{k}^{-1}\dpd{\v{S}_k}{\theta_i}\v{S}_{k}^{-1}
	\label{eq:K_pd}\\
	\dpd{\v{m}_{k|k}}{\theta_i}&=\dpd{\v{m}_{k|k-1}}{\theta_i}+\dpd{\v{K}_k}{\theta_i}\left(\v{y}_k-\v{H}\v{m}_{k|k-1}\right)-\v{K}_k\v{H}\dpd{\v{m}_{k|k-1}}{\theta_i}
	\label{eq:m_pd}\\
	\dpd{\v{P}_{k|k}}{\theta_i}&=\dpd{\v{P}_{k|k-1}}{\theta_i}-\dpd{\v{K}_k}{\theta_i}\v{S}_{k}\v{K}_{k}^\tr-\v{K}_{k}\dpd{\v{S}_k}{\theta_i}\v{K}_{k}^\tr-\v{K}_{k}^\tr\v{S}_{k}\left(\dpd{\v{K}_k}{\theta_i}\right)^\tr
	\label{eq:P_pd}
	\end{align}
Equations \eqref{eq:m_pred_pd}, \eqref{eq:P_pred_pd}, \eqref{eq:K_pd}, \eqref{eq:m_pd} and \eqref{eq:P_pd} together specify
a recursive algorithm for computing \eqref{eq:dlogLH} that can be run alongside the Kalman filter recursions.
As noted in \textcite{Cappe2005}, these equations are sometimes known as the \emph{sensitivity equations}
and they are derived at least in \textcite{Gupta1974} and \textcite{Mbalawataa}.
\todo{Elaborate on nonlinear programming}  

%%%%%%%%%%%%%%%%%%%%%%%%%%%%%%%%%%%%%%%%%%%%%%
\subsection{Expectation maximization (EM)}%%%%
%%%%%%%%%%%%%%%%%%%%%%%%%%%%%%%%%%%%%%%%%%%%%%

The expectation maximization (EM) algorithm \parencite{Dempster1977} is a general
method for finding ML and MAP estimates in probabilistic models with missing data or
latent variables \parencite{Bishop2006,barber2012bayesian}. As will be seen, instead of maximizing
\eqref{eq:logLH} directly, EM alternates between forming a variational lower bound and maximizing it.
We shall use $\E{\cdot}{q}\equiv\defint{}{}{\cdot \Pdf[q]{z}}{z}$ to denote the expectation
over some arbitrary distribution $\Pdf[q]{z}$.
Let us introduce a ``variational'' 
distribution $\tPX$ over the states, parameterized with $\Th'$ (not necessarily related to $\Th$).
Toting now that $\Pdf{\X}{\Y,\Th}=\cLH/\LH$ and that $\lLH\equiv\log\LH$ is independent of $\X$ we can then perform the
following decomposition on the log likelihood:
\begin{align}
	\lLH &= \log\cLH - \log\post \nonumber\\
	&= \E{\log\cLH}_{\tP} - \E{\log\post}_{\tP} \nonumber\\
	&= \underbrace{\E{\log\cLH-\log\tPX}_{\tP}}_{\LB[1pt]{\tP,\Th}}
	\underbrace{-\E{\log\post-\log\tPX}_{\tP}}_{\KL{\tP}{\post[1pt]}}
	\label{eq:lLH_decomp}
\end{align}
The important step here is taking the expectation over $\tPX$, since the \emph{complete-data log-likelihood}
$\log\cLH$ cannot be evaluated as $\X$ is unobserved.
Since $\KL{\tP}{\post}$, the \emph{Kullback-Leibler divergence} between $\tPX$ and $\post$, is always nonnegative,
we see that 
\begin{align}
	\lLH&\geq \LB{\tP,\Th} \label{eq:em_lh_lb}
\end{align}
with equality when 
\begin{align}
	\tPX &= \post, \label{eq:em_maxq}
\end{align}
i.e. the posterior distribution of the states with equal parameter value $\Th'=\Th$. Considered as a functional
of only $\tP$, clearly $\LB{\tP,\Th}$ is maximized
and $\KL{\tP}{\post}$ vanishes by \eqref{eq:em_maxq}. 
The nonnegativeness of the Kullback-Leibler divergence can be proved by
noting that $-\log$ is a convex function and so \emph{Jensen's inequality}
can be applied \parencite{Bishop2006}.

Let us take a closer at the the first term in \eqref{eq:lLH_decomp} with $\tPX=\Pdf{\X}{\Y,\Th'}$:
\begin{align}
	\LB{\Pdf{\X,\Y}{\Th'},\Th}&=
	\underbrace{\E{\log\cLH}_{\Pdf[p][1pt]{\X}{\Y,\Th'}}}_{\F[1pt]{\Lb}{\Th',\Th}} 
	-\E{\log\Pdf{\X,\Y}{\Th'}}_{\Pdf[p][1pt]{\X}{\Y,\Th'}}.
	\label{eq:completedata_loglikelihood}
\end{align}
Clearly the latter term (the differential entropy of $\Pdf{\X,\Y}{\Th'}$) is constant 
with respect to $\Th$, so that maximizing $\mathcal{L}$ with respect to $\Th$ amounts to
maximizing $\F{\Lb}{\Th',\Th}$, the \emph{expected complete-data log-likelihood},
with respect to $\Th$.


We are now ready the define the EM algorithm, which produces
a series of estimates $\{\Th_j\}$ to the parameter $\Th$
starting from an initial guess $\Th_0$. The two alternating
steps of the algorithm are:

\begin{description}
\addtolength{\leftskip}{1cm}
  \item[E-step]\hfill\\
  Given the current estimate $\Th_j$ of the parameters, compute	
  	\begin{align}
		\tP_{j+1} &= \argmax_{\tP}\LB{\tP,\Th_j}.
		\label{eq:EM_E}
	\end{align}
  As stated, the maximum is obtained with 
  $\tP_{j+1} = \Pdf{\X}{\Y,\Th_j}$, the posterior
  distribution of the states given the current parameter estimate. After the maximization
  we have
  	\begin{align}
  		\lLH[\Th_j]&=\LB{\Pdf{\X}{\Y,\Th_j},\Th_j} \label{eq:EM_E_lh}\\
  		 &= \F{\Lb}{\Th_j,\Th_j}+\mathtt{const}\nonumber 
	\end{align}
  \item[M-step]\hfill\\ 
  Set
    \begin{align}
		\Th_{j+1}&=\argmax_{\Th}\LB{\tP_{j+1},\Th} \label{eq:EM_M}\\
		&=\argmax_{\Th}\F{\Lb}{\Th_j,\Th}\nonumber.
	\end{align}
\end{description}
We are now in a position to formulate the so called \emph{fundamental inequality of EM} \parencite{Cappe2005}:
\begin{align}
	\lLH[\Th_{j+1}] - \lLH[\Th_j]\geq & \LB{\Pdf{\X}{\Y,\Th_j},\Th_{j+1}} - \LB{\Pdf{\X}{\Y,\Th_j},\Th_{j}} 
	\label{eq:fundamental_inequality}
\end{align}
which is just the combination of \eqref{eq:em_lh_lb} and \eqref{eq:EM_E_lh}. But it highlights
the fact that \emph{the likelihood is increased or unchanged with every new estimate} $\Th_{j+1}$.
Also following from \eqref{eq:fundamental_inequality} is the fact that if the iterations
stop at a certain point, i.e. $\Th_{l+1}=\Th_l$ at iteration $l$, then
$\LB{\Pdf{\X}{\Y,\Th_l},\Th}$ must be maximal at $\Th_l$
and so the gradients of the lower bound and of the likelihood must be zero. Thus
$\Th_l$ is a \emph{stationary point} of $\lLH$, i.e a local maximum or a saddle point.


\subsubsection{EM as a special case of variational Bayes}
\todo{Ensure that notation matches prev chapter}

\parencite{barber2012bayesian,jordan1998learning}
Variational Bayes (VB) is a fully Bayesian methodology where one seeks
for an approximation to the parameter posterior
\begin{align}
	\Pdf{\Th}{\Y}\propto \defint{\mathcal{X}}{}{\Pdf{\X,\Y}{\Th}}{\X}\Pdf{\Th}
	\label{tablelabel}
\end{align}
As mentioned earlier, finding this distribution is commonly intractable, so in VB
we assume a factorized form for the joint posterior of states and parameters
\begin{align}
	\Pdf{\X,\Th}{\Y}\approx \Pdf[q]{\X}\Pdf[q]{\Th}
	\label{eq:VB_factorization}
\end{align}
and the task is then to find the best approximation with respect
to the KL divergence between the true posterior and the approximation
\begin{align}
	\KL{\Pdf[q]{\X}\Pdf[q]{\Th}}{\Pdf{\X,\Th}{\Y}} &= \E{\Pdf[q]{\X}}{\Pdf[q]{\X}} + \E{\Pdf[q]{\X}}{\Pdf[q]{\X}} -
	\E{\Pdf{\X,\Th}{\Y}}{\Pdf[q]{\X}\Pdf[q]{\Th}}.
	\label{eq:KL_VB}
\end{align}
Using ${\Pdf{\X,\Th}{\Y}}=\Pdf{\X,\Th,\Y}/\Pdf{\Y}$ equation \eqref{eq:KL_VB} gives
\begin{align}
	\lLH &\geq \E{\Pdf{\X,\Th,\Y}}{\Pdf[q]{\X}\Pdf[q]{\Th}} - \E{\Pdf[q]{\X}}{\Pdf[q]{\X}} -
	\E{\Pdf[q]{\X}}{\Pdf[q]{\X}}
	\label{eq:VB_bound}
\end{align}
and thus minimizing the KL divergence is equivalent to finding the tightest lower bound to
the log likelihood. Analogously to EM, minimizing the KL divergence is done iteratively
keeping $\Pdf[q]{\Th}$ fixed and minimizing w.r.t $\Pdf[q]{\X}$ in the ``E''-step
and vice versa in the ``M''-step:

\begin{description}
\addtolength{\leftskip}{1cm}
\item[E-step]
\begin{align}
	\Pdf[q^{\text{new}}]{\X}=\argmax_{\Pdf[q]{\X}}\left(\KL{\Pdf[q]{\X}\Pdf[q^{\text{old}}]{\Th}}{\Pdf{\X,\Th}{\Y}}\right)
	\label{eq:VB_E}
\end{align}
\item[M-step]
\begin{align}
	\Pdf[q^{\text{new}}]{\Th}=\argmax_{\Pdf[q]{\Th}}\left(\KL{\Pdf[q^{\text{new}}]{\X}\Pdf[q]{\Th}}{\Pdf{\X,\Th}{\Y}}\right)
	\label{eq:VB_M}
\end{align}
\end{description}


Let us then suppose that we only wish to find the MAP point estimate $\Th^*$. This can be accomplished
by assuming a delta function form $\Pdf[q]{\Th}=\delta\left(\Th,\Th^*\right)$ for the parameter factor in the
joint distribution of states and parameters \eqref{eq:VB_factorization}.
With this assumption equation \eqref{eq:VB_bound} becomes
\begin{align}
	\Pdf{\Y}{\Th^*} &\geq \E{\Pdf{\X,\Th^*,\Y}}{\Pdf[q]{\X}\Pdf[q]{\Th}} - \E{\Pdf[q]{\X}}{\Pdf[q]{\X}} + \mathtt{const}
	\label{eq:VB_MAP_boundl}
\end{align}
and the ``M''-step \eqref{eq:VB_M} can then be written as
\begin{align}
	\Th^* &= \text{argmax}_{\Th}\left(\E{\log\cLH}{\Pdf[q]{\X}}+\log\Pdf{\Th}\right).
	\label{tablelabel}
\end{align}
If the point estimate is plugged in the ``E''-step equation \eqref{eq:VB_E} we have
\begin{align}
	\Pdf[q^{\text{new}}]{\X}\propto \Pdf{\X,\Y}{\Th^*} \propto \Pdf{\X}{\Y,\Th^*} 
	\label{tablelabel}
\end{align}

\subsubsection{Partial E and M steps}


%%%%%%%%%%%%%%%%%%%%%%%%%%%%%%%%%%
\subsection{EM in SSMs}%%%%%%%%%%
%%%%%%%%%%%%%%%%%%%%%%%%%%%%%%%%%%
\label{sec:EM_SSM}

Let us then look at how to apply EM to a SSM of the form \eqref{eq:ssm_general}. First of
all, from the factorization in \eqref{eq:complete_data_likelihood}, the complete-data log-likelihood becomes
\begin{align*}
\begin{split}
	\cLH =&-\frac{1}{2}\left(\x_0-\gv{\mu}_0\right)^\tr\gv{\Sigma}_0^{-1}\left(\x_0-\gv{\mu}_0\right)-\frac{1}{2}\log\abbs{\gv{\Sigma}_0}\\
	&-\frac{1}{2}\sum_{k=1}^T\left(\x_k-\v{f}(\x_{k-1})\right)^\tr\v{Q}^{-1}\left(\x_k-\v{f}(\x_{k-1})\right)-\frac{T}{2}\log\abbs{\v{Q}}\\
	&-\frac{1}{2}\sum_{k=1}^T\left(\y_k-\v{h}(\x_{k})\right)^\tr\v{R}^{-1}\left(\y_k-\v{h}(\x_{k})\right)-\frac{T}{2}\log\abbs{\v{R}}\\
	&+\mathtt{const}
\end{split}
\end{align*}
Taking the expectation w.r.t. $\Pdf{\X}{\Y,\Th'}$ (assumed implicitly in the notation), applying the identity $\v{a}^\tr\v{C}\v{b}=\Tr{\v{a}^\tr\v{C}\v{b}}=\Tr{\v{C}\v{b}\v{a}^\tr}$, 
denoting $\f_{k-1}\equiv \f(\x_{k-1})$ and $\h_k\equiv \h(\x_k)$ and dropping the constant terms we get
\begin{align}
\begin{split}
	\F{\Lb}{\Th',\Th} =&-\frac{1}{2}\Tr{\gv{\Sigma}_0^{-1}\E{\left(\x_0-\gv{\mu}_0\right)\left(\x_0-\gv{\mu}_0\right)^\tr}}-\frac{1}{2}\log\abbs{\gv{\Sigma}_0}\\
	&-\frac{1}{2}\Tr{\v{Q}^{-1}\sum_{k=1}^T\E{\left(\x_k-\f_{k-1}\right)\left(\x_k-\f_{k-1}\right)^\tr}}-\frac{T}{2}\log\abbs{\v{Q}}\\
	&-\frac{1}{2}\Tr{\v{R}^{-1}\sum_{k=1}^T\E{\left(\y_k-\h_{k}\right)\left(\y_k-\h_{k}\right)^\tr}}-\frac{T}{2}\log\abbs{\v{R}}
\end{split}
\label{eq:eclLH}
\end{align} 
Let us denote the three expectations in equation~\eqref{eq:eclLH} with
\begin{align}
	\v{I}_1 &= \E{\left(\x_0-\gv{\mu}_0\right)\left(\x_0-\gv{\mu}_0\right)^\tr}\\ 
	&=\defint{\mathcal{X}\times T}{}{\left(\x_0-\gv{\mu}_0\right)\left(\x_0-\gv{\mu}_0\right)^\tr\Pdf{\X}{\Y,\Th'}}{\X}\nonumber\\
	&= 	\defint{\mathcal{X}}{}{\left(\x_0-\gv{\mu}_0\right)\left(\x_0-\gv{\mu}_0\right)^\tr\Pdf{\x_0}{\Y,\Th'}}{\x_0} \label{eq:I1_general}\\
	\v{I}_2 &= \sum_{k=1}^T\defint{\mathcal{X}\times 2}{}{\left(\x_k-\f_{k-1}\right)\left(\x_k-\f_{k-1}\right)^\tr\Pdf{\bm{\xk^\tr&\xkk^\tr}^\tr}{\Y,\Th'}}{\bm{\xk&\xkk}^\tr} \label{eq:I2_general}\\
	\v{I}_3 &= \sum_{k=1}^T\defint{\mathcal{X}}{}{\left(\y_k-\h_{k}\right)\left(\y_k-\h_{k}\right)^\tr\Pdf{\xk}{\Y,\Th'}}{\xk} \label{eq:I3_general}
\end{align}
It is clear then that in the E-step one needs to compute the $T+1$ smoothing
distributions, including the $T$ cross-timestep distributions, since these
will be needed in the expectations.
By applying the identity
\begin{align}
	\var{\x}&=\E{\x\x^\tr}-\E{\x}\E{\x}^\tr,
\end{align} 
we can already write the first expectation as
\begin{align}
	\v{I}_1&= \v{P}_{0|T}+(\v{m}_{0|T}-\gv{\mu}_0)(\v{m}_{0|T}-\gv{\mu}_0)^\tr.
	\label{eq:I1}
\end{align}
This was a result of assuming the Gaussian prior distribution of equation~\eqref{eq:prior} and
it is common for the more specific models we're assessing in the next chapters.

Since in this thesis we're assuming Gaussian noise, the M-step maximization equations for
the complete noise covariance matrices $\v{Q}$ and $\v{R}$ will also be common.
To derive these equations, we will derivate \eqref{eq:eclLH} with respect to these
matrices. As can be seen from \eqref{eq:eclLH}, the terms involving $\v{Q}$ or $\v{R}$
are analogous and so we will only the derivation for $\v{Q}$ and only the result
for $\v{R}$. It's easier to take the derivative with respect to $\v{Q}^{-1}$:
\begin{align}
	\dpd{\Lb(\Th',\Th)}{\v{Q^{-1}}}=
	&-\frac{1}{2}\dpd{}{\v{Q^{-1}}}\Tr{\v{Q}^{-1}\sum_{k=1}^T\E{\left(\xk-\f_{k-1})\right)\left(\xk-\f_{k-1})\right)^\tr}}\nonumber\\
	&-\frac{T}{2}\dpd{}{\v{Q^{-1}}}\log\abbs{\v{Q}}\nonumber\\
	=&-\frac{1}{2}\sum_{k=1}^T\E{\left(\xk-\f_{k-1})\right)\left(\xk-\f_{k-1})\right)^\tr}+\frac{T}{2}\v{Q},
	\label{eq:clLH_pdQ}
\end{align}
where we have used formula 92 in \cite{Petersen2008} for the first derivative and 
formula 51 for the second. Setting \eqref{eq:clLH_pdQ}  
to zero \todo{Why is it same for the inverse?} we get the update equation for the next estimate of $\v{Q}$
\begin{align}
	\v{Q}_{j+1}&=\frac{1}{T}\sum_{k=1}^T\E{\left(\xk-\f_{k-1}\right)\left(\xk-\f_{k-1}\right)^\tr} \label{eq:EM_M_Q}\\
	&=\frac{1}{T}\v{I}_2\nonumber
\end{align}
The derivation of the update equation for the next estimate of $\v{R}$ is exactly analogous, giving
\begin{align}
	\v{R}_{j+1}&=\frac{1}{T}\sum_{k=1}^T\E{\left(\yk-\h_{k}\right)\left(\yk-\h_{k}\right)^\tr} \label{eq:EM_M_R}\\
	&=\frac{1}{T}\v{I}_3\nonumber
\end{align} 
In the following more specific cases, we will only
consider the two remaining expectations and the rest of the M-step maximization
equations.

Let us then discuss a certain important aspect of the M-step maximization equations, with a consequence
that is useful also for gradient based methods for ML. Often times
one is not interested in estimating any \emph{full} matrix, with which we mean that the model
might have been parameterized so that there are only scalar parameters embedded inside matrices or 
as part of $\f$ or $\h$. In that case
it is quite likely that the M-step maximization equations will not admit themselves to closed form
analytical expressions. Rather the use of nonlinear optimization methods is needed and for that
we would like have the gradient of $\Lb$. This is fortunately rather straightforward. Let
$\theta_i$ be an element of $\Th$. Then
\begin{align}
\label{eq:dLB_nonlinear}
\begin{split}
	\dpd{\Lb(\Th,\Th')}{\theta_i}
	=\quad{}&\frac{1}{2}\mathrm{Tr}\left[\gv{\Sigma}^{-1}\left(
	\dpd{\gv{\Sigma}}{\theta_i}\gv{\Sigma}^{-1}
	\sum_{k=1}^T\E{\left(\x_0-\gv{\mu}_0\right)\left(\x_0-\gv{\mu}_0\right)^\tr}{\hat{\Th}_j}\right.\right.\\
	&\quad+2\sum_{k=1}^T\E{\dpd{\gv{\mu}_0}{\theta_i}\left(\x_0-\gv{\mu}_0\right)^\tr}
	-\left.\left.\dpd{\gv{\Sigma}}{\theta_i}\vphantom{\sum_{k=1}^T}\right)\right]\\
	+&\frac{1}{2}\mathrm{Tr}\left[\v{Q}^{-1}\left(\dpd{\v{Q}}{\theta_i}\v{Q}^{-1}
	\sum_{k=1}^T\E{\left(\x_k-\f_{k-1}\right)\left(\x_k-\f_{k-1}\right)^\tr}\right.\right.\\
	&\quad +2\sum_{k=1}^T\E{\dpd{\f_{k-1}}{\theta_i}\left(\x_k-\f_{k-1}\right)^\tr}
	-\left.\left. T\dpd{\v{Q}}{\theta_i}\vphantom{\sum_{k=1}^T}\right)\right]\\
	+&\frac{1}{2}\mathrm{Tr}\left[\v{R}^{-1}\left(\dpd{\v{R}}{\theta_i}\v{R}^{-1}
	\sum_{k=1}^T\E{\left(\y_k-\h_{k}\right)\left(\y_k-\h_{k}\right)^\tr}\right.\right.\\
	&\quad +2\sum_{k=1}^T\E{\dpd{\h_{k}}{\theta_i}\left(\y_k-\h_{k}\right)^\tr}
	-\left.\left.T\dpd{\v{R}}{\theta_i}\vphantom{\sum_{k=1}^T}\right)\right].
\end{split}	
\end{align}

What is interesting about this quantity is that it can also be used to compute the gradient of the log-likelihood, i.e
the score, itself. From equation~\eqref{eq:lLH_decomp} it can be seen rather easily, that  
the partial derivative of the log-likelihood evaluated at $\widehat{\Th}$ is given by
\begin{align}
		\left.\dpd{\lLH}{\theta_i}\right|_{\Th=\widehat\Th}&=
		\left.\dpd{\LB{\Pdf{\X}{\Y,\widehat\Th},\Th}}{\theta_i}\right|_{\Th=\widehat\Th}=
		\left.\dpd{ \F{\Lb}{\widehat\Th,\Th}}{\theta_i}\right|_{\Th=\widehat\Th} \label{eq:EM_gradients}
\end{align} 
Equation \eqref{eq:EM_gradients} is known as \emph{Fisher's identity} \parencite{Cappe2005}. It gives
an alternative route for the score function computation.
\todo{elaborate on score computation}

%%%%%%%%%%%%%%%%%%%%%%%%%%%%%%%%%%%%%%%%%%%%%%%%%%
\subsubsection{EM in linear-Gaussian SSMs}%%%%%%%
%%%%%%%%%%%%%%%%%%%%%%%%%%%%%%%%%%%%%%%%%%%%%%%%%%
\parencite{shumway1982approach,Ghahramani1996}
Let us substitute $\v{A}\xkk$ for $\f_{k-1}$ and $\v{H}\xk$ for $\h_{k}$.
Let us also denote by
\begin{align}
	\Pdf{\xk, \xkk}{\v{Y},\Th}&=
	\N[
	\begin{bmatrix}
		\xk\\\xkk
	\end{bmatrix}
	]{\m_{k,k-1|T},\P_{k,k-1|T}} \label{eq:pdf_smooth_joint},
\end{align}
the joint smoothing distribution of $\xk$ and $\xkk$.
Then by applying the manipulation
\begin{align}
\begin{split}
&\E{\left(\v{x}_k-\v{A}\v{x}_{k-1}\right)\left(\v{x}_k-\v{A}\v{x}_{k-1}\right)^\tr}\\
=&\bm{\v{I} & -\v{A}}	
\E{
\begin{bmatrix}
	\xk\\\xkk
\end{bmatrix}
\begin{bmatrix}
	\xk^\tr & \xkk^\tr	
\end{bmatrix}
}
\bm{\v{I}\\-\v{A}^\tr}	
\end{split}
\end{align}
we get
\begin{align}
	\v{I}_{2}&=
\bm{\v{I} & -\v{A}}	
\sum_{k=1}^T\left(\P_{k,k-1|T}+\m_{k,k-1|T}\m_{k,k-1|T}^\tr\right)
\bm{\v{I}\\-\v{A}^\tr} \label{eq:LGSSM_I2}\\
&=\bm{\v{I} & -\v{A}}	
%\bm{\P_{k|T}+\m_{k|T}\m_{k|T}^\tr & \left(\P_{k,k-1|T}+\m_{k,k-1|T}\m_{k,k-1|T}^\tr\right)^\tr \\ \P_{k,k-1|T}+\m_{k,k-1|T}\m_{k,k-1|T}^\tr & \P_{k-1|T}+\m_{k-1|T}\m_{k-1|T}^\tr }
\bm{\sum_{k=1}^T\E{\xk\xk^\tr} & \sum_{k=1}^T\E{\xk\xkk^\tr} \\ \sum_{k=1}^T\E{\xkk\xk^\tr} & \sum_{k=1}^T\E{\xkk\xkk^\tr} }
\bm{\v{I}\\-\v{A}^\tr}\nonumber\\
&=\bm{\v{I} & -\v{A}}	
\bm{\bar{\X}_{00} & \bar{\X}_{01} \\ \bar{\X}_{01}^\tr & \bar{\X}_{11} }
\bm{\v{I}\\-\v{A}^\tr}\nonumber\\
&=\bar{\X}_{00}-\v{A}\bar{\X}_{01}^\tr-\bar{\X}_{01}\v{A}^\tr+\v{A}\bar{\X}_{11}\v{A}^\tr\nonumber\\
&=\left(\v{A}-\bar{\X}_{01}\bar{\X}_{11}^{-1}\right)\bar{\X}_{11}\left(\v{A}-\bar{\X}_{01}\bar{\X}_{11}^{-1}\right)^\tr+\bar{\X}_{00}+\bar{\X}_{01}\bar{\X}_{11}^{-1}\bar{\X}_{01}^\tr.
\label{eq:Amax}
\end{align}
It's easy to see that the extremum value of the last line with respect to $\v{A}$
is obtained by setting
\begin{align}
	\v{A}_{j+1}&=\bar{\X}_{01}\bar{\X}_{11}^{-1} \label{eq:EM_M_A}.	
\end{align}
For $\v{I}_3$ we get
\begin{align}
\v{I}_{3}&=
\bm{\v{I} & -\v{H}}	
\sum_{k=1}^T
\bm{\yk\yk^\tr & \yk\E{\xk}^\tr \\ \E{\xk}\yk^\tr & \E{\xk\xk^\tr} }
\bm{\v{I}\\-\v{H}^\tr} \label{eq:LGSSM_I3}\\
&=\bm{\v{I} & -\v{H}}	
\sum_{k=1}^T
	\bm{\bar{\Y}_{00} & \bar{\v{C}}_{00} \\ \bar{\v{C}}_{00}^\tr & \bar{\X}_{00} }
\bm{\v{I}\\-\v{H}^\tr},
\end{align}
so with analogous manipulation as in \eqref{eq:Amax} we get
\begin{align}
	\v{H}_{j+1}&=\bar{\v{C}}_{00}\bar{\X}_{00}^{-1} \label{eq:EM_M_H}.	
\end{align}

All in all, the E-step of EM algorithm in linear-Gaussian SSM:s consists of
computing the $T$ joint distributions in equation~\eqref{eq:pdf_smooth_joint} with the RTS smoother.
Actually this is not the only option, as for example in
\textcite{Elliott1999} a new kind of filter is presented that
can compute the expectations with only forward recursions.
After this, the M-step estimates are computed for $\v{Q}$
from equation~\eqref{eq:EM_M_Q}, for $\v{R}$ from equation~\eqref{eq:EM_M_R}, for $\v{A}$ from equation~\eqref{eq:EM_M_A}
and for $\v{H}$ from equation~\eqref{eq:EM_M_H}. Also for evaluating the convergence,
one needs to compare the current value of $\Lb$ with the previous one.
For this we can compute $\v{I}_1$ from equation~\eqref{eq:I1},
$\v{I}_2$ from equation~\eqref{eq:LGSSM_I2} and $\v{I}_3$\todo{index $\v{I}$:s} from equation~\eqref{eq:LGSSM_I3}.

For working with structured matrices let us rewrite the gradient equation~\eqref{eq:dLB_nonlinear}
in the linear-Gaussian case:
\begin{align}
\label{eq:dLB_linear}
\begin{split}
	\dpd{\Lb(\Th,\Th')}{\theta_i}
	=\quad{}&\frac{1}{2}\mathrm{Tr}\left[\gv{\Sigma}^{-1}\left(
	\dpd{\gv{\Sigma}}{\theta_i}\gv{\Sigma}^{-1}
	\sum_{k=1}^T\E{\left(\x_0-\gv{\mu}_0\right)\left(\x_0-\gv{\mu}_0\right)^\tr}{\hat{\Th}_j}\right.\right.\\
	&\quad+2\sum_{k=1}^T\E{\dpd{\gv{\mu}_0}{\theta_i}\left(\x_0-\gv{\mu}_0\right)^\tr}
	-\left.\left.\dpd{\gv{\Sigma}}{\theta_i}\vphantom{\sum_{k=1}^T}\right)\right]\\
	+&\frac{1}{2}\mathrm{Tr}\left[\v{Q}^{-1}\left(\dpd{\v{Q}}{\theta_i}\v{Q}^{-1}\bm{\v{I} & -\v{A}}\bm{\bar{\X}_{00} & \bar{\X}_{01} \\ \bar{\X}_{01}^\tr & \bar{\X}_{11} }\bm{\v{I}\\-\v{A}^\tr}\right.\right.\\
	&\quad +2\bm{\v{0} & -\dpd{\v{A}}{\theta_i}}\bm{\bar{\X}_{00} & \bar{\X}_{01} \\ \bar{\X}_{01}^\tr & \bar{\X}_{11} }\bm{\v{I}\\-\v{A}^\tr}
	-\left.\left. T\dpd{\v{Q}}{\theta_i}\vphantom{\sum_{k=1}^T}\right)\right]\\
	+&\frac{1}{2}\mathrm{Tr}\left[\v{R}^{-1}\left(\dpd{\v{R}}{\theta_i}\v{R}^{-1}
	\bm{\v{I} & -\v{H}}\bm{\bar{\Y}_{00} & \bar{\v{C}}_{00} \\ \bar{\v{C}}_{00}^\tr & \bar{\X}_{00} }\bm{\v{I}\\-\v{H}^\tr}\right.\right.\\
	&\quad +2\bm{\v{0} & -\dpd{\v{H}}{\theta_i}}\bm{\bar{\Y}_{00} & \bar{\v{C}}_{00} \\ \bar{\v{C}}_{00}^\tr & \bar{\X}_{00} }\bm{\v{I}\\-\v{H}^\tr}
	-\left.\left.T\dpd{\v{R}}{\theta_i}\vphantom{\sum_{k=1}^T}\right)\right].
\end{split}
\end{align}


%%%%%%%%%%%%%%%%%%%%%%%%%%%%%%%%%%%%%%%%%%%%%%%%%%%
\subsubsection{EM in nonlinear-Gaussian SSMs}%%%%%
%%%%%%%%%%%%%%%%%%%%%%%%%%%%%%%%%%%%%%%%%%%%%%%%%%%

As explained in section \ref{sec:nonlinear_state}, in then nonlinear case
the filtering and smoothing distributions cannot be computed exactly.
Thus the E-step is also only approximate and the convergence
guarantees of EM won't apply anymore. The proposed methods can nevertheless
provide good results most of the time. In the fortunate case that the
model is linear-in-the-parameters the M-step can be solved in closed form.
This situation will be covered later in section~\ref{sec:litp}. Currently we will assume
however that the model is nonlinear in the parameters as well as in the states so that
the simplest form to write the model is exactly \eqref{eq:ssm_general}.

The E-step in the EM-algorithm consists now of computing approximations to $\v{I}_1$, $\v{I}_2$ and $\v{I}_3$
from equations~\eqref{eq:I1}, \eqref{eq:I2_general} and \eqref{eq:I3_general}. In the
M-step one should a gradient based nonlinear optimization algorithm to compute
the next parameter estimate \eqref{eq:EM_M}. The gradient can be computed with
the equation~\eqref{eq:dLB_nonlinear}.  


%%%%%%%%%%%%%%%%%%%%%%%%%%%%%%%%%%%%%%%%%%%%%%%%%%%%%%%%%
\subsubsection{EM in linear-in-the-parameters SSM:s}%%%%%
%%%%%%%%%%%%%%%%%%%%%%%%%%%%%%%%%%%%%%%%%%%%%%%%%%%%%%%%%
\label{sec:litp}

Suppose now that $\v{f}(\x):\mathcal{X}\to\mathcal{X}$ is a linear
combination of vector valued functions $\gv{\rho}_k:\mathcal{X}\to\R^{d_{\Phi,k}}$,
so that the parameters of $\f$, $\gv{\Phi}_k$, are matrices of size $d_x\times d_{\Phi,k}$.
Then $\f$ can be written as 
\begin{align}
\begin{split}
	\v{f}(\x)&=\gv{\Phi}\gv{\rho}_1(\x)+\dots+\gv{\Phi}_m\gv{\rho}_m(\x)\\
	&=
	\begin{bmatrix}
		\gv{\Phi}_1 & \dots & \gv{\Phi}_m
	\end{bmatrix}
	\begin{bmatrix}
		\gv{\rho}_1(\x)\\
		\vdots\\ 
		\gv{\rho}_m(\x)
	\end{bmatrix}\\
	&=\v{A}\v{g}(\x),
\end{split}
\end{align}
so that $\v{A}$ is now a matrix of size ${d_x\times\sum_{k=1}^m d_{\Phi,k}}$ and $\v{g}:\mathcal{X}\to \R^{\sum_{k=1}^m d_{\Phi,k}}$ . 
Denoting $\v{g}\left(\xkk\right)\equiv\v{g}_{k-1}$ and $\v{l}\left(\xk\right)\equiv\v{l}_{k}$
and following the derivation in equation~\eqref{eq:LGSSM_I2} we have
\begin{align}
	\v{I}_{2}
&=\bm{\v{I} & -\v{A}}	
\bm{\sum_{k=1}^T\E{\xk\xk^\tr} & \sum_{k=1}^T\E{\xk\v{g}_{k-1}^\tr} \\ \sum_{k=1}^T\E{\v{g}_{k-1}\xk^\tr} & \sum_{k=1}^T\E{\v{g}_{k-1}\v{g}_{k-1}^\tr} }
\bm{\v{I}\\-\v{A}^\tr}\nonumber\\
&=\bm{\v{I} & -\v{A}}	
\bm{\bar{\X}_{00} & \bar{\v{G}}_{01} \\ \bar{\v{G}}_{01}^\tr & \bar{\v{G}}_{11} }
\bm{\v{I}\\-\v{A}^\tr}
\end{align}
Then similarly to \eqref{eq:EM_M_A}
\begin{align}
	\v{A}_{j+1}&=\bar{\v{G}}_{01}\bar{\v{G}}_{11}^{-1} \label{eq:EM_LIP_M_A}.	
\end{align}

Applying similar assumptions as for $\v{f}$ to $\h$, we can write
\begin{align}
\begin{split}
	\v{h}(\x)&=\gv{\Upsilon}\gv{\pi}_1(\x)+\dots+\gv{\Upsilon}_m\gv{\pi}_m(\x)\\
	&=
	\begin{bmatrix}
		\gv{\Upsilon}_1 & \dots & \gv{\Upsilon}_m
	\end{bmatrix}
	\begin{bmatrix}
		\gv{\pi}_1(\x)\\
		\vdots\\ 
		\gv{\pi}_m(\x)
	\end{bmatrix}\\
	&=\v{H}\v{l}(\x),
\end{split}
\end{align}
where the size of $\v{H}$ is now ${d_x\times\sum_{k=1}^m d_{\Upsilon,k}}$ and 
$\v{l}:\mathcal{X}\to \R^{\sum_{k=1}^m d_{\Upsilon,k}}$ .
For $\v{I}_3$ we get
\begin{align}
\v{I}_{3}&=
\bm{\v{I} & -\v{H}}	
\sum_{k=1}^T
\bm{\yk\yk^\tr & \yk\E{\v{l}_k}^\tr \\ \E{\v{l}_k}\yk^\tr & \E{\v{l}_k\v{l}_k^\tr} }
\bm{\v{I}\\-\v{H}^\tr} \label{eq:LGSSM_I3}\\
&=\bm{\v{I} & -\v{H}}	
\sum_{k=1}^T
	\bm{\bar{\Y}_{00} & \bar{\v{D}}_{00} \\ \bar{\v{D}}_{00}^\tr & \bar{\v{L}}_{00} }
\bm{\v{I}\\-\v{H}^\tr},
\end{align}
and
\begin{align}
	\v{H}_{j+1}&=\bar{\v{D}}_{00}\bar{\v{L}}_{00}^{-1} \label{eq:EM_LIP_M_H}.	
\end{align}


\subsection{Properties of the estimation algorithms}

Both the direct method of section~\ref{sec:grad} and the EM algorithm of section~\ref{sec:EM_SSM}
have their strengths and weaknesses. Neither of them can be said to eclipse the other in an absolute sense.
In this section we will go through some features of the said algorithms found in the literature.
Then a more detailed analysis will be performed on two essential aspects of any estimation algorithm:
the convergence properties, i.e. when should one expect the algorithm to find a local maximum and
the computational requirements of the algorithms.

\textcite{Cappe2005} contains a list of arguments in favor of either of the methods. The following list includes
those and additional points with comments:
\begin{description}
  \item[Direct]\hfill
\begin{itemize}
  \item\emph{No smoother needed} The log-likelihood, or an approximation to it, can be evaluated
  with forward-filtering
  \item\emph{No M-step}. There is no need to figure out model-dependent maximization
  formulas even in nonlinear models 
  \item\emph{Faster convergence}. Advanced gradient-based optimization
  methods can reach convergence speeds that are close to quadratic
\end{itemize}
  \item[EM]\hfill
  \begin{itemize}
  \item \emph{Simple to implement}. This argument is often put forward in favor of the EM
 algorithm. However in practice when using the gradient-based method one would use any one of
the off-the-self nonlinear optimizers and not re-implement one. Thus which one is easier to implement
boils down to gradient computation. If the model is linear or linear-in-the-parameters the EM
algorithm doesn't need any gradient information.
  \item\emph{Parameter constraints}. In \textcite{Cappe2005} it is argued that
since the M-step maximization equations are so simple, including parameter constraints
are easier in the EM algorithm. This again depends on if the model is linear or linear-in-the-parameters.
  \item\emph{Parameterization independent}. This again depends on if one has to use gradient-based
 optimization in the M-step. If not, then the EM algorithm is parameterization independent. In the gradient-based
 method the gradient and the Hessian, and so the convergence, are affected by the parameterization.
\end{itemize} 
\end{description}
 

\subsubsection{Limit points}

\subsubsection{Convergence rates}

Let us denote the mapping implicitly defined by the EM
algorithm by $\v{M}:\Theta\to\Theta$. Then if the iterates
$\Th_j$ converge to a point $\Th_\star$, we must have $\Th_\star=\F{\v{M}}{\Th_\star}$.

\parencite{Wu1983,Sandell1978,Meng1997,Elliott1999,Salakhutdinov2003a,Salakhutdinov2003,Olsson2007,Paninski2010}

\subsubsection{Computational load}

\parencite{Harvey1990,Watson1983,Cappe2005,Saatci2011,Olsson2007,Salakhutdinov2003a}


