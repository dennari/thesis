\subsubsection{Ascent methods}

An important theoretical result concerning iterative ascent methods is known
as the \emph{global convergence theorem} \parencite{luenberger2008}. It states
the necessary conditions that the estimate produced by an iterative ascent method
is a local maximum. Suppose then that $\ell:\Theta\to\R$ is the objective function
and we are trying to solve the \emph{unconstrained} optimization problem
\begin{align}
	\Th_\star&=\argmax_{\Th}\lLH,\quad\Th\in\Theta=\R^{d_\theta},
	\label{eq:optimization_problem}
\end{align}
where $\Th_\star$ need not be unique, i.e. there can be a solution set 
\begin{align}
	\mathcal{S}=\brac*{\Th_\star\in\Theta : \lLH[\Th_\star]\geq \lLH[\Th] \; \forall \Th\in\Theta}
\end{align}
To formulate the conditions under which this problem can be solved by an iterative ascent method,
we need to define the concept of a closed point-to-set mapping. $T$ is a point-to-set map
from $X$ to $Y$, if it maps points in $X$ to subsets of $Y$.
$T$ is said to be \emph{closed} at $\x \in X$ if
\begin{alignat*}{3}
	\mathrm{(i)}&\; & \x_j & \to \x,&\;&\x_j \in X,\;j=0,1,2,\dots\\
	\mathrm{(ii)} && \y_j &\to \y, && \y_j\in\F{T}{\x_j},\;j=0,1,2,\dots\\
\shortintertext{imply that}
	&& \y&\in\F{T}{\x}	&&
\end{alignat*}
This can be considered a generalization of continuity for point-to-set mappings.
\begin{theorem}
Given an initial value $\Th_0$, let the point-to-set map $\EMM*:\Theta\to Y$ construct a sequence
of estimates $\brac[\big]{\Th_{j+1}\in\Theta : \Th_{j+1}\in\EMM{\Th_j}}$, let $\mathcal{S}\subset\Theta$ denote the solution
set and let $\ell:\Theta\to\R$ denote the ``ascent'' (or objective) function. 
Then given the conditions
\begin{enumerate}[i)] \addtolength{\leftskip}{1cm} \itemsep1pt \parskip0pt \parsep0pt
	\item all $\Th_j$ belong to $\Theta_c$,  a compact subset of $\Theta$
	\item $\lLH$ is a continuous function in $\Theta$ with
\begin{eqspace}{5pt}{5pt}{1pt}{5pt}
\begin{align*}
	\Th_j \not\in \mathcal{S} \Rightarrow \lLH[\Th_{j+1}] > \lLH[\Th_j]\\  
	\Th_j \in \mathcal{S} \Rightarrow \lLH[\Th_{j+1}] \geq \lLH[\Th_j]  
\end{align*}
\end{eqspace}
	\item $\EMM*$ is closed at $\Theta\setminus\mathcal{S}$
\end{enumerate}
it must follow that the limit of every subsequence of $\brac*{\Th_j}$ is part of $\mathcal{S}$ 
\end{theorem}
As pointed out in \textcite{luenberger2008}, it is usually the third condition
of the above theorem that produces difficulties with actual implementations
of iterative ascent algorithms. We will discuss this point in further detail in connection
with the specific methods we're considering.


\subsubsection{Properties of the estimators}
Among different point estimates, the maximum likelihood estimator has good statistical properties.
Let us denote the true parameter value, the value that the data was generated with, with $\Th_\star$ and 
let $T$ denote the amount of observations.
Then provided that some conditions of not very restricting nature hold, we can state the following asymptotic properties 
for the ML estimate $\Th_{\text{ML}}$:\todo{Modify to reflect p.465 in Cappé}
\begin{description}
\addtolength{\leftskip}{1cm}
\item[Strong consistency]\hfill\\
An important property for an estimator, which says that
the estimator tends to the true value as the amount of data tends to infinity:
\begin{align}
	%\forall \Th \in \Theta\quad \frac{1}{n}\Pdf[\ell_n]{\Th} \xrightarrow{\mathrm{a.s.}} \lLH, \mathrm{when} n\to\infty
	\F{\ell_T}{\Th_{\text{ML}}} \xrightarrow{\mathrm{a.s.}} \F{\ell}{\Th_\star},\quad \mathrm{when}\;T\to\infty,
\end{align}
where $\ell_T$ is the likelihood function after $T$ measurements and $\ell$ is a continuous
deterministic function with a unique global maximum at $\Th_\star$.
%where $\Pdf[\ell_n]{\Th}$ is the log-likelihood given $n$ observations and $\lLH$ is a continuous deterministic
%function with a unique global maximum at $\Th_\star$.
\item[Asymptotic normality]\hfill\\
This property gives us the means to compute asymptotic error bounds for
the estimate:
\begin{align}
	\sqrt{T}\left(\Th_{\text{ML}}-\Th_\star\right) \xrightarrow{D} \N{\v{0}}{\F{\mathcal{I}^{-1}}{\Th_\star}},\quad \mathrm{when}\;T\to\infty,	
\end{align}
where $\F{\mathcal{I}}{\Th_\star}$ is the \emph{Fisher information matrix} evaluated at $\Th_\star$ 
\item[Efficiency]\hfill\\
When the amount of information tends to infinity, the ML-estimate achieves
the Cramér-Rao lower bound, i.e. no other consistent estimator has lower asymptotic mean-squared-error.
\end{description}



\subsubsection{Identifiability}

Intuitively, any parameters $\Th,\:\Th' \in \Theta,\; \Th \not= \Th'$ cannot be distinguished
from each other with maximum likelihood estimation if
\begin{align}
	\Pdf{\Y}{\Th}&=\Pdf{\Y}{\Th'},
\end{align}
i.e., if the same data can arise with two (or more) separate
parameter values. Let us then go through the second order sufficient conditions
for a point $\Th_\star$ to be a local maximum. We can define a local maximum
to be a point $\Th_\star$, for which $\lLH[\Th_\star]-\lLH[\Th_\star+\v{d}] \geq 0$ for small $\v{d}$.

\begin{proposition}\label{prop:cond_for_max}
Let $\ell:\Theta \to \R$ have continuous second order partial derivatives and let $\Th_\star$
belong to the interior of $\Theta$. If
\begin{enumerate}[i)] \addtolength{\leftskip}{1cm} \itemsep1pt \parskip0pt \parsep0pt
  \item $\nabla\lLH[\Th_\star] =0$ 
  \item $\nabla^2\lLH[\Th_\star]$ is negative definite
\end{enumerate}
then $\Th_\star$ is a strict local maximum of $\ell$.
\end{proposition}
 \begin{proof}
 Since \textit{ii)}, there must exist a nonnegative number $a$ for which 
 $-\v{d}^\tr\nabla^2\lLH[\Th_\star]\v{d} \geq a\abs{\v{d}}^2$.
 Taylor expanding around $\Th_\star$ gives
 \begin{align*}
	\lLH[\Th_\star+\v{d}] &= \lLH[\Th_\star]+\nabla\lLH[\Th_\star]^\tr\v{d}+\frac{1}{2}\v{d}^\tr\nabla^2\lLH[\Th_\star]\v{d}+\F[\big]{o}{\,\abs{\v{d}}^2}\\
	\lLH[\Th_\star]-\lLH[\Th_\star+\v{d}] &= -\frac{1}{2}\v{d}^\tr\nabla^2\lLH[\Th_\star]\v{d}-\F[\big]{o}{\,\abs{\v{d}}^2}\\
	&\geq \frac{1}{2}a\abs{\v{d}^2}-\F[\big]{o}{\,\abs{\v{d}}^2}.
\end{align*}
For small enough $\abs{\v{d}}$, the right hand side of the last row must be nonnegative.
 \end{proof}

\subsubsection{Global convergence}

We would like to demonstrate the applicability of theorem~\ref{th:global_convergence}.
Let $\EMM*$ then denote the point-to-set map implicitly defined by the EM algorithm.
From theorem~\ref{th:global_convergence} it can be deduced, that the convergence of all the
subsequences of $\brac[\big]{\Th_j}$ to some point in $\mathcal{S}$ implies that
the objective function converges to some value $\ell_\star$, which is the same
for all $\Th_\star\in\mathcal{S}$.

Suppose now that 
\begin{enumerate}[i)] \addtolength{\leftskip}{1cm} \itemsep1pt \parskip0pt \parsep0pt
	\item we are given the initial value $\Th_0$
	\item the level set $\Theta^0=\brac{\Th\in\Theta : \lLH[\Th]\geq \Th_0}$ 
is compact and contained in the interior of $\Theta$
	\item the solution set is $\mathcal{L}^0=\brac*{\Th\in\Theta^0 : \nabla\lLH[\Th]=0}$
	\item $\EMH$ is continuous in both of its arguments
	\item $\lLH[\Th]\geq\lLH[\Th']\quad \forall \Th\in\EMM{\Th'}$
\end{enumerate}
Then $\lim_{j\to\infty}\Th_j\in\mathcal{L}^0$ and 
this also applies to every subsequence of $\brac*{\Th_j}$. Additionally
$\lim_{j\to\infty}\lLH[\Th_j]=\lLH[\Th_\star]$ where $\Th_\star\in\mathcal{L}^0$.
For the parameter sequence to converge to some 
$\Th_\star\in\mathcal{L}^0$, we need to assume in addition that $\lim_{j\to\infty}\abs{\Th_{j+1}-\Th_{j}}=0$

We will now assume that $\EMM$ is singular, i.e.
$\v{M}:\Theta\to\Theta$, and continuous. Then if $\brac[\big]{\Th_0,\EMM{\Th_0},\EMM{\EMM{\Th_0}},\dots }$
converges to a point $\Th_\star$, we must have $\Th_\star=\EMM{\Th_\star}$ and
thus around $\Th_\star$ we obtain the first order Taylor series expansion
\begin{align}
	\Th_{j+1}-\Th_\star&=\EMM{\Th_j}-\EMM{\Th_\star}\approx\dif\EMM{\Th_\star}\left(\Th_j-\Th_\star\right).
\end{align} 
We can then arrive at the following expression for $\dif\EMM{\Th_\star}$ \parencite{Dempster1977,Gibson2005,Lange1995}:
\begin{align}
	\dif\EMM{\Th_\star} &= \v{I}-\mathcal{I}_{\X\Y}^{-1}\mathcal{I}_{\Y}\\
	&=\mathcal{I}_{\X}\mathcal{I}_{\X\Y}^{-1}
	\label{eq:dM},
\end{align}
where
\begin{align}
		\mathcal{I}_{\X\Y} = \dQ &= \eval{\E{\prtdd{\log\cLH}{\Th}{\Th}}^{-1}_{\post{\Th_\star}}}_{\Th=\Th_\star}\\
		\mathcal{I}_{\Y} = \dL &= \eval{\E{\prtdd{\log\LH}{\Th}{\Th}}_{\post{\Th_\star}}}_{\Th=\Th_\star}\\
		\mathcal{I}_{\X} = \dH &= \eval{-\E{\prtdd{\log\post}{\Th}{\Th}}_{\post{\Th_\star}}}_{\Th=\Th_\star}
\end{align}
The latter term in equation~\eqref{eq:dM} is commonly interpreted as the proportion of ``measured'' information compared to
all information. All the eigenvalues of $\dif\EMM{\Th_\star}$ belong to $[0,1)$ and the \emph{linear} 
rate of convergence near $\Th_\star$ is the largest of them \parencite{Lange1995}. Then
clearly the rate of convergence is determined by how closely $\mathcal{I}_{\Y}$ ``resembles''
$\mathcal{I}_{\X\Y}$, i.e the more information is ``hidden'' in the hidden variables $\X$ the
slower is the convergence.\todo{Take Salakhutdinov2003 in here}   

There is still the problem that points in $\mathcal{L}^0$ can in addition
to being local maxima, be also saddle points or even local minima. In order to formulate
the conditions that guarantee that the stationary point is a local maximum, let us assume
from now on that the identifiability issues have been solved and $\EMM$ is thus a point-to-point map.
Then if $\mathcal{I}_{\X\Y}$ and $\mathcal{I}_{\X}$ are positive definite,
$\dif\EMM{\Th_\star}$ has only positive real eigenvalues and $\Th_\star$ is a
stable maximizer if and only if $\mathcal{I}_{\Y}$ has only negative eigenvalues.

\todo{What about Petersen2005?}
In \textcite{Salakhutdinov2004,Salakhutdinov2003} the transformation matrix $\F{\mathcal P}{\Th}$ is defined as
\begin{align}
	\Th_{j+1}-\Th_j &= \F{\mathcal P}{\Th_j}\nabla\lLH[\Th_j],
	\label{eq:EM_P}
\end{align}
then near $\Th_\star$ (i.e when $j$ is large)
we get
\begin{align}
		\F{\mathcal P}{\Th_j}&\approx -\left(\v{I}-\dif\EMM{\Th_j}\right)\dL^{-1}.
		\label{eq:EM_P_approx}
\end{align}
What we can infer from \eqref{eq:EM_P} and \eqref{eq:EM_P_approx} is that, near $\Th_\star$, if the dominant eigenvalue
of $\dif\EMM$ approaches unity, i.e. when the proportion of hidden information approaches zero, the EM algorithm behaves like 
a true Newton's method. On the other hand when the dominant 
eigenvalue of $\dif\EMM$ approaches zero, i.e when the proportion of hidden information approaches unity, the convergence 
of the EM-algorithm becomes slower and stops. 

\todo{Note the fast initial convergence}

\parencite{Wu1983,Sandell1978,Meng1997,Elliott1999,Olsson2007,Paninski2010}

\subsubsection{Local convergence}

\subsubsection{Complexity}

\parencite{Harvey1990,Watson1983,Cappe2005,Saatci2011,Olsson2007}

