\documentclass[english,12pt]{article}

% LAYOUT
\usepackage{mylayout}
\setmainlanguage{english}

% replaces Helvetica in the cover page
\newfontfamily\texgyreheros[
	Path=/usr/share/texmf/fonts/opentype/public/tex-gyre/,
    Extension=.otf,
    UprightFont= *-regular,
    BoldFont=*-bold,
    ItalicFont=*-italic,
    BoldItalicFont=*-bolditalic,
    Mapping={tex-text}
]{texgyreheros}

\usepackage{aaltothesis}



% PDF SETUP
\usepackage[unicode,bookmarks, colorlinks, breaklinks,
pdftitle={Dippa},
pdfauthor={Ville Väänänen},
pdfproducer={xetex}
]{hyperref}
\hypersetup{linkcolor=black,citecolor=black,filecolor=black,urlcolor=MidnightBlue}
%\usepackage{mcode}

%\usepackage[shadow]{todonotes}
% MATH
\usepackage{mymath}
%\everymath{\displaystyle}


\newcommand{\Th}{\gv{\theta}}
\newcommand{\LH}{\Pdf{\v{Y}}{\Th}}
\newcommand{\LHf}[1]{\Pdf{\v{Y}}{#1}}
\newcommand{\LHh}{\Pdf{\v{Y}}{\hat{\Th}}}
\newcommand{\lLH}{L\!\left(\Th\right)}
\newcommand{\cLH}{\Pdf{\v{X},\v{Y}}{\Th}}
\newcommand{\lcLH}{\log\cLH}
\newcommand{\LB}{\mathcal{L}}
\newcommand{\Lb}{\mathfrak{L}}
\newcommand{\KL}[2]{\mathrm{KL}\left(#1\|#2\right)}


% BIBLIOGRAPHY
\usepackage[hyperref=true,backend=biber]{biblatex} % TEXLIPSE BUG: backend cannot be first
\addbibresource{Dippa.bib}


\setlength{\hoffset}{-1in}
\setlength{\oddsidemargin}{35mm}
\setlength{\evensidemargin}{25mm}
\setlength{\textwidth}{15cm}
\setlength{\voffset}{-1in}
\setlength{\headsep}{7mm}
\setlength{\headheight}{1em}
\setlength{\topmargin}{25mm-\headheight-\headsep}
\setlength{\textheight}{23cm}

\begin{document}

%% Korjaa vastaamaan korkeakouluasi
%%
%% Change the school field to describe your school 
\university{aalto university}{aalto-yliopisto}
\school{School of Electrical Engineering}{Sähkötekniikan korkeakoulu}

%% Vain kandity�lle: Korjaa seuraavat vastaamaan tutkinto-ohjelmaasi
%%
%% Only for B.Sc. thesis: Choose your degree programme. 
\degreeprogram{Electronics and electrical engineering}%
{Elektroniikka ja sähkötekniikka}
%%

%% Vain DI/M.Sc.- ja lisensiaatinty�lle: valitse laitos, 
%% professuuri ja sen professuurikoodi. 
%%
%% Only for M.Sc. and Licentiate thesis: Choose your department,
%% professorship and professorship code. 
\department{Department of Biomedical Engineering and Computational Science}%
{Lääketieteellisen tekniikan ja laskennalisen tieteen laitos}
\professorship{Computational and Cognitive Biosciences}{Laskennallinen ja
kognitiivinen biotiede}
\code{S-114}
%%

%% Valitse yksi n�ist� kolmesta
%%
%% Choose one of these:
\univdegree{MSc}

%% Oma nimi
%%
%% Should be self explanatory...
\author{Ville Väänänen}

%% Opinn�ytteen otsikko tulee vain t�h�n. �l� tavuta otsikkoa ja
%% v�lt� liian pitk�� otsikkoteksti�. Jos latex ryhmittelee otsikon
%% huonosti, voit joutua pakottamaan rivinvaihdon \\ kontrollimerkill�.
%% Muista ett� otsikkoja ei tavuteta! 
%% Jos otsikossa on ja-sana, se ei j�� rivin viimeiseksi sanaksi 
%% vaan aloittaa uuden rivin.
%% 
%% Your thesis title. If the title is very long and the latex 
%% does unsatisfactory job of breaking the lines, you will have to
%% break the lines yourself with \\ control character. 
%% Do not hyphenate titles.
\thesistitle{Gaussian filtering and smoothing based parameter estimation in nonlinear models for sequential data}
{Gaussiseen suodatukseen ja siloitukseen perustuva parametrien estimointi epälineaarisissa aikasarjamalleissa}

\place{Espoo}
%% Kandidaatinty�n p�iv�m��r� on sen esitysp�iv�m��r�! 
%% 
%% For B.Sc. thesis use the date when you present your thesis. 
\date{\today}

%% Kandidaattiseminaarin vastuuopettaja tai diplomity�n valvoja.
%% Huomaa titteliss� "\" -merkki pisteen j�lkeen, 
%% ennen v�lily�nti� ja seuraavaa merkkijonoa. 
%% N�in tehd��n, koska kyseess� ei ole lauseen loppu, jonka j�lkeen tulee 
%% hieman pidempi v�li vaan halutaan tavallinen v�li.
%%
%% B.Sc. or M.Sc. thesis supervisor 
%% Note the "\" after the comma. This forces the following space to be 
%% a normal interword space, not the space that starts a new sentence. 
\supervisor{Prof.\ Jouko Lampinen}{Prof.\ Jouko Lampinen}

%% Kandidaatinty�n ohjaaja(t) tai diplomity�n ohjaaja(t)
%% 
%% B.Sc. or M.Sc. thesis instructor(s)
%\instructor{Prof. Pirjo Professori}{Prof. Pirjo Professori}
\instructor{D.Sc.\ (Tech.) Simo Särkkä}{TkT Simo Särkkä}
%\instructor{M.Sc.\ (Tech.) Polli Pohjaaja}{DI Polli Pohjaaja}

%% Aaltologo: syntaksi: \uselogo{red|blue|yellow}{?|!|''}
%% Logon kieli on sama kuin dokumentin kieli
%%
%% Aalto logo: syntax: \uselogo{red|blue|yellow}{?|!|''} 
%% Logo language is set to be the same as the document language.
\uselogo{aaltoRed}{''}

%% Tehd��n kansilehti
%%
%% Create the coverpage

\makecoverpage


%
%% English abstract, uncomment if you need one. 
%% 
%% Abstract keywords
\keywords{Parameter estimation, Sequential data, Nonlinear state space models,
Expectation maximization, Quasi--Newton optimization}
%% Abstract text
\begin{abstractpage}[english]
State space modeling is a widely used statistical approach
for sequential data. Commonly the resulting models can be considered to contain
two interconnected estimation problems: that of the dynamic states
and that of the static parameters. The difficulty of these problems
depends critically on the linearity of the model, either with
respect to the states, the parameters or both.\\\\%
%
In this thesis we show how to obtain maximum likelihood and maximum a posteriori
estimates for the static parameters. Two methods are considered: gradient based nonlinear
optimization of the marginal log-likelihood and expectation maximization.
The former requires the filtering distributions and the latter both the
filtering and the smoothing distributions.
We show how the efficient Gaussian filtering based methods
can be applied to obtain these distributions when the model
is nonlinear.\\\\%
%
The resulting optimization equations are demonstrated in a linear model
with simulated data and a nonlinear model with actual photoplethysmograph
data. 

\end{abstractpage}

\newpage
%% Note that 
%% if you are writting your master's thesis in English place the English
%% abstract first followed by the possible Finnish abstract

%% Suomenkielinen tiivistelmä
%% 
%% Finnish abstract
%%
%% Tiivistelmän avainsanat
\keywords{Parametrien estimointi, Aikasarjat, Epälineaariset tila-avaruusmallit,
EM, Kvasi--Newton optimointi}
%% Tiivistelmän tekstiosa
\begin{abstractpage}[finnish]
Tila-avaruusmallinnus on eräs laajalti käytetty aikasarjojen mallinnusmenetelmä.
Tila-avaruusmallin voidaan ajatella sisältävän kaksi keskenään
vuorovaikkuteista estimointiongelmaa: dynaamisten tilojen estimointi
sekä staattisten parametrien estimointi. Näiden estimointiongelmien
vaikeuteen vaikuttaa erityisen paljon mallin lineaarisuus -- sekä
tilojen että parametrien suhteen.\\\\%
%
Tässä diplomityössä näytämme, kuinka 
suurimman uskottavuuden estimaattori, jonka voidaan ajatella olevan
erikoistapaus a posteriori tiheysfunktion maksimoivasta estimaattorista, 
voidaan johtaa tila-avaruusmallin staattisille parametreille.
Vertailemme kahta eri menetelmää: 
uskottavuusfunktion gradienttipohjaista epälineerista optimointia
sekä expectation maximization algoritmiä.\\\\%
%
Lopputuloksina saatuja optimointialgoritmeja sovelletaan kahdessa
eri tapauksessa, joista toisessa käytetään lineaarista
mallia ja simuloitua dataa ja toisessa epälineaarista mallia
ja oikeaa mittalaitteesta peräisin olevaa dataa.


\end{abstractpage}

%% Pakotetaan uusi sivu varmuuden vuoksi, jotta 
%% mahdollinen suomenkielinen ja englanninkielinen tiivistelmä
%% eivät tule vahingossakaan samalle sivulle
%%
%% Force new page so that English abstract starts from a new page
%




%\newpage


\addcontentsline{toc}{section}{Contents}
\tableofcontents

%%% Symbolit ja lyhenteet
%%
%% Symbols and abbreviations
%\mysection{Symbolit ja lyhenteet}
%\subsection*{Symbolit}
\subsection*{Notation}
\begin{tabular}{ll}
$\v{Z}$  & Matrix (bold uppercase letter)  \\
$\v{z}$ & Column vector (bold lowercase letter) \\
$\v{z}_{1:T}$    & set of vectors $\brac{\v{z}_1,\dots,\v{z}_T}$
\end{tabular}


\subsection*{Symbols}

\begin{tabular}{ll}
$\Th$            & Parameter\\
$\Pdf{\x}{\y}$   & Probability density function of $\x$ conditional on $\y$
\end{tabular}

\subsection*{Abbreviations}

\begin{tabular}{ll}
SSM & State space model \\
MAP & Maximum a posteriori \\
ML & Maximum likelihood \\
EM & Expectation maximization \\
AR & Autoregressive \\
fMRI & Functional magnetic resonance imaging \\
MEG & Magnetoencephalography \\
HMM & Hidden Markov model \\
DAG & Directed acyclic graph \\
PDF & probability density function \\
RTS & Rauch--Tung--Striebel \\
SMC & Sequential Monte Carlo \\
GHKF & Gauss--Hermite Kalman filter \\
CKF & Cubature Kalman filter \\
CKS & Cubature Kalman smoother \\
EKF & Extended Kalman filter \\
PMCMC & Particle Markov chain Monte Carlo \\
BFGS & Broyden--Fletcher--Goldfarb--Shanno quasi--Newton update\\
VB & Variational Bayes \\
vEM & Variational expectation maximization \\
ECG & Expectation--conjugate--gradient
\end{tabular}


%% Sivulaskurin viilausta opinnäytteen vaatimusten mukaan:
%% Aloitetaan sivunumerointi arabialaisilla numeroilla (ja jätetään
%% leipätekstin ensimmäinen sivu tyhjäksi, 
%% ks. alla \thispagestyle{empty}).
%% Pakotetaan lisäksi ensimmäinen varsinainen tekstisivu alkamaan 
%% uudelta sivulta clearpage-komennolla. 
%% clearpage on melkein samanlainen kuin newpage, mutta 
%% flushaa myös LaTeX:n floatit 
%% 
%% Corrects the page numbering, there is no need to change these
\cleardoublepage
\storeinipagenumber
\pagenumbering{arabic}
\setcounter{page}{1}


%% Leipäteksti alkaa
%%
%% Text body begins. Note that since the text body
%% is mostly in Finnish the majority of comments are
%% also in Finnish after this point. There is no point in explaining
%% Finnish-language specific thesis conventions in English.
\section{Introduction}

%% Ensimmäinen sivu tyhjäksi
%% 
%% Leave first page empty
\thispagestyle{empty}

This thesis is about parameter estimation
in dynamical systems\ldots


%% Opinnäytteessä jokainen osa alkaa uudelta sivulta, joten \clearpage
%%
%% In a thesis, every section starts a new page, hence \clearpage
\clearpage

\section{State-space models}

In this report we will be focusing on models that
are defined in so called \emph{state space} form.
The state is denoted with $\v{x}$, and it belongs to the
space $\R^d$.
More specifically the models we are considering can be written as

These equations can be understood as specifying a model,
where at every step $k$ the state $\v{x}_k$ is observed
by a noisy observation (or measurement) $\v{y}_k$. Furthermore the states
evolve as specified by the equation \eqref{eq:dynamics}.
Thus the states form a \emph{Markov chain}. 

\subsection{Linear}

More specifically the models we are considering can be written as
\begin{subequations}
	\label{eq:the_model}
	\begin{align}
		\v{x}_k&=\v{A}\v{x}_{k-1}+\v{q}_{k-1} \label{eq:dynamics}\\
		\v{y}_k&=\v{H}\v{x_k}+\v{r}_k \label{eq:measurements}\\
		\v{q}_{k-1} &\sim \N{0,\v{Q}} \label{eq:pdf_dynamic_noise}\\
		\v{r}_{k} &\sim \N{0,\v{R}} \label{eq:pdf_measurement_noise}\\
		\v{x}_{0} &\sim \N{\gv{\mu},\gv{\Sigma}}
	\end{align}
\end{subequations}
Equations \eqref{eq:dynamics}, \eqref{eq:measurements}, \eqref{eq:pdf_dynamic_noise} and \eqref{eq:pdf_measurement_noise}
together specify the following conditional distributions
\begin{align}
		\v{x}_{k}|\v{x}_{k-1} &\sim \N{\v{A}\v{x}_{k-1},\v{Q}} \label{eq:pdf_xk_xk-1}\\
		\v{y}_{k}|\v{x}_{k} &\sim \N{\v{H}\v{x_k},\v{R}}  \label{eq:pdf_yk_xk}
\end{align}

Linearity in this case means that $\v{x}_k$ is a linear combination
of the elements of $\v{x}_{k-1}$ and $\v{y}_k$ is a linear combination
of the elements of $\v{x}_{k}$ (with additive noise in both cases). Since
the noise terms $\v{q}_{k-1}$ and $\v{r}_{k}$ are assumed to be white and
Gaussian, these models are called linear-Gaussian.

\subsection{Nonlinear}


The SSM model is now
\begin{subequations}
	\label{eq:the_model}
	\begin{align}
		\v{x}_k&=\v{f}(\v{x}_{k-1})+\v{q}_{k-1} \label{eq:dynamics_nonlinear}\\
		\v{y}_k&=\v{h}(\v{x_k})+\v{r}_k \label{eq:measurements_nonlinear}\\
		\v{q}_{k-1} &\sim \N{0,\v{Q}} \label{eq:pdf_dynamic_noise_nonlinear}\\
		\v{r}_{k} &\sim \N{0,\v{R}} \label{eq:pdf_measurement_noise_nonlinear}\\
		\v{x}_{0} &\sim \N{\gv{\mu},\gv{\Sigma}}
	\end{align}
\end{subequations}
We assume an implicit dependence of $f$ and $h$ on the parameter $\Th$.

\subsection{Linear in the parameters}

Suppose the function $\v{f}$ is linear in the parameters and the dimension of the state $\v{x}$ is $d$.
Then in the most general case $\v{f}(\v{x}):\R^d\to\R^d$ is a linear
combination of vector valued functions $\gv{\rho}_k(\v{x}):\R^d\to\R^{d_k}$  
and the parameters are matrices $\gv{\Phi}_k\in\R^{d\times d_k}$. More specifically,
we have

\begin{align}
\begin{split}
	\v{f}(\v{x})&=\gv{\Phi}\gv{\rho}_1(\v{x})+\dots+\gv{\Phi}_m\gv{\rho}_m(\v{x})\\
	&=
	\begin{bmatrix}
		\gv{\Phi}_1 & \dots & \gv{\Phi}_m
	\end{bmatrix}
	\begin{bmatrix}
		\gv{\rho}_1(\v{x})\\
		\vdots\\ 
		\gv{\rho}_m(\v{x})
	\end{bmatrix}\\
	&=\v{A}\v{g}(\v{x}),
\end{split}
\end{align}
where $\v{A}\in\R^{d\times\sum_{k=1}^m d_k}$ and $\v{g}(\v{x}):\R^d\to \R^{\sum_{k=1}^m d_k}$ . 
For example, in case of the function
\begin{align}
	f(x,t)&=ax+b\frac{x}{1+x^2}+c\cos(1.2t)
\end{align}
we would have
\begin{align}
	f(x,t)&=
	\begin{bmatrix}
		a & b & c
	\end{bmatrix}
	\begin{bmatrix}
		x\\
		\frac{x}{1+x^2}\\ 
		\cos(1.2t)
	\end{bmatrix}
\end{align}
Suppose now, that the matrix $A$ depends on parameters $\v{s}$.



\section{Optimal filtering and smoothing}

\subsection{Kalman filtering}
\subsection{RTS smoothing}
\subsection{Example}
\subsection{Sigma point methods}
\subsubsection{Unscented Kalman filter}
\subsubsection{Gauss-Hermite Kalman filter}
\subsubsection{Cubature Kalman filter}
\subsection{Example}
\subsection{Particle filtering}

\section{Parameter estimation}

\section{Nonlinear optimization}
\subsection{Marginal likelihood}
\subsection{Gradient}

\section{Expectation Maximization}






\end{document}

