\subsection{State space models}

State space models (SSMs) provide a unified probabilistic methodology for modeling
sequential data \parencite{ljung1994modeling,durbin2012time,Cappe2005,barber2011bayesian}. 
Sequential data arise in numerous applications, 
typically in the form of time-series measurements. Modern time-series data arise 
often in the context of medical imaging,
for example in the case of functional magnetic resonanse imaging (fMRI) or magnetoencephalography (MEG).
However it is not necessary for the sequence index to have
a temporal meaning. In probabilistic terms, a time-series
can be described by a \emph{stochastic process} $\y = \left\{\y(t): t\in \mathcal{T}\right\}$, 
where $\y(t)$ is a random variable and $\mathcal{T}\subseteq\R$ for continuous time or 
$\mathcal{T}\subseteq\field{N}$ for discrete time sequences.
In this thesis we will only be concerned with discerete time processes 
and we write $\y_{1:k}\equiv\{\y_1,\dots,\y_k\}\equiv\{\y(t_1),\dots,\y(t_k)\}$.

A fundamental question in probabilistic models for sequential data is how 
to model the dependence between variables. It is infeasible to assume
that every random variable in the process depends on all the others.
Thus it is common to assume a \emph{Markov chain}, where the distribution of
the process at the current timestep depends only on the probability distribution in the previous timestep.
A further assumption in SSMs is that the process of interest, the dynamic process $\x$, is not directly observed
but only through another stochastic process, the \emph{measurement process} $\y$. Since
$\x$ is not observed, SSMs belong to the class of \emph{latent variable models}. Sometimes, as in
\textcite{Cappe2005}, SSMs are called \emph{hidden Markov models} (HMM) but usually this implies that
the sample space of $\x$ is discrete. Yet another term for a quite general subclass
of SSMs is \emph{dynamic Bayesian networks} (DBNs). These and some connections
to classical time-series modeling approaches are discussed in \textcite{Murphy2002}. 

An important characteristic of SSMs is that the values of the measurement process are conditionally independent
given the latent Markov process. An intuitive way to present conditional 
independence properties between random variables is a \emph{Bayes network} 
presented by a directed acyclic graph (DAG) \parencite{pearl1988probabilistic,Bishop2006}.
A Bayes network presentation of a discrete-time SSM is given in Figure~\ref{fig:ssm_graphical}.

\begin{figure}[!htb]
	\centering
	\begin{tikzpicture}
	\tikzstyle{main}=[circle, minimum size = 5mm, inner sep=0mm, thick, draw =black!80, node distance = 12mm,font=\small]
	\tikzstyle{param}=[circle, minimum size = 2mm, inner sep=0mm, thick, draw =white!80, fill=black!80, node distance = 18mm]
	\tikzstyle{ellipsis}=[circle, minimum size = 10mm, thick, draw =black!80, node distance = 18mm]
	\tikzstyle{connect}=[-latex, semithick]
	\tikzstyle{tconnect}=[-latex, thin, opacity=0.25]
	  \node[main] (x_1) [label=above:$\v{x}_{k-1}$] {};
	  \node[main,draw=black!0] (prev) [left of=x_1] {$\dots$};
	  \node[main] (x1) [left of=prev,label=above:$\v{x}_{1}$] {};
	  \node[main] (x) [right of=x_1,label=above:$\v{x}_{k}$] {};
	  \node[main] (x__1) [right of=x,label=above:$\v{x}_{k+1}$] {};
	  \node[main,fill = black!10] (y_1) [below of=x_1,label=below:$\v{y}_{k-1}$] {};
	  \node[main,fill = black!10] (y) [below of=x,label=below:$\v{y}_{k}$] {};
	  \node[main,fill = black!10] (y__1) [below of=x__1,label=below:$\v{y}_{k+1}$] {};
	  \node[main,fill = black!10] (y1) [below of=x1,label=below:$\v{y}_{1}$] {};
	  \node[main,draw=white!0] (next) [right of=x__1] {$\dots$};
	  \node[main] (xT) [right of=next,label=above:$\v{x}_{T}$] {};
	  \node[main,fill = black!10] (yT) [below of=xT,label=below:$\v{y}_{T}$] {};
	  \node[main,node distance=15mm] (theta) [above of=x,label=above:$\Th$] {};
	  \path
	    (x1) edge [connect] (prev)
	    	  edge [connect] (y1)
	    (prev) edge [connect] (x_1)
	    (x_1) edge [connect] (x) 
	    	  edge [connect] (y_1)
	    (x) edge [connect] (y)
	    	edge [connect] (x__1)
	    (x__1) edge [connect] (y__1)
	    	edge [connect] (next)
	    (next) edge [connect] (xT)
	    (xT) edge [connect] (yT)
	    (theta) edge [tconnect] (x__1)
	    	edge [tconnect] (x)
	    	edge [tconnect] (x_1)
	    	edge [tconnect] (x1)
	    	edge [tconnect] (xT)
	    	edge [tconnect] (y1)
	    	edge [tconnect] (y_1)
	    	edge [tconnect,bend left=20] (y)
	    	edge [tconnect] (y__1)
	    	edge [tconnect] (yT);
	\end{tikzpicture}
	\caption{A discrete-time state space model as a graphical model presented with a directed acyclic graph.
	Each node represents a random variable and arrows present dependence. The hidden variables $\xk$, meaning the
	states, form a Markov chain and each state has a corresponding measurement $\yk$, which is oberved. Given the states, the measurements
	are independent. Both the states and the measurements depend on the parameter $\Th$.}
	\label{fig:ssm_graphical}
\end{figure}%
The value $\xk \in \mathcal{X} \equiv \R^{d_x}$ of the dynamic process at time $t_k$ is called the
\emph{state} at time $t_k$. As explained in the introduction, the state summarizes as much information
about the dynamic process as is needed to formulate the dynamic model introduced below. 
For the measurements we define $\yk \in \mathcal{Y} \equiv \R^{d_y}$.
As depicted in Figure~\ref{fig:ssm_graphical}, we assume that the joint \emph{probability density function} 
(PDF, will be used interchangeably with \emph{density} and \emph{distribution}) of
$\X$ and $\Y$ is conditional on a set of parameters $\Th \in \Theta \subseteq \R^{d_\theta}$. 
%Then the problem of (Bayesian) state estimation can be stated as learning the distribution of
%$\X$ given $\Y$ and the problem of parameter estimation as learning the distribution
%of $\Th$ given $\Y$.
 
Taking into account the Markov property
\begin{align}
	\Pdf{\xk}{\x_{1:k-1},\Th}&=\Pdf{\xk}{\x_{k-1},\Th}
\end{align}
of the dynamic process and the conditional
independence property 
\begin{align}
	\Pdf{\yk}{\x_{1:k},\y_{1:k-1},\Th}&=\Pdf{\yk}{\xk,\Th}
\end{align}
of the measurement process, the joint density of states
and measurements factorises as
\begin{align}
	\Pdf{\X,\Y}{\Th}&=\Pdf{\v{x}_0}{\Th}
	\prod_{k=1}^T\Pdf{\v{x}_k}{\v{x}_{k-1},\Th}
	\prod_{k=0}^T\Pdf{\v{y}_k}{\v{x}_{k},\Th}
	\label{eq:complete_data_likelihood}.
\end{align}
Thus in order to describe a SSM one needs to specify three distributions:
\begin{description}
\addtolength{\leftskip}{1cm}
	\item[Prior distribution]
	$\Pdf{\v{x}_0}{\Th}$ is the distribution assumed for the state prior to observing any measurements. The
	sensitivity of the marginal posterio distribution to the prior depends 
	on the amount of data (the more data the less sensitivity).
	\item[Dynamic model]
	$\Pdf{\v{x}_k}{\v{x}_{k-1},\Th}$ dictates the time evolution of the states.
	\item[Measurement model]
	$\Pdf{\v{y}_k}{\v{x}_{k},\Th}$ models how the observations depend on the state and the statistics of the noise.
\end{description}
In this thesis it is assumed that the parametric form of these distributions is known
for example by physical modeling \parencite{ljung1994modeling}. Regarding the notation,
we will overload $\Pdf{\cdot}{\cdot}$ as a generic probability density function
specified by its arguments. Also the difference between random variables and their realizations is suppressed.

Traditionally SSMs are specified as a pair of equations specifying the dynamic and measurement models. 
In great generality, discrete-time SSMs can be described by the following dynamic and measurement equations 
\begin{subequations}
\label{eq:ssm_too_general}
\begin{align}
	\xk &= \F{\v{f}_{k}}{\x_{k-1},\v{q}_{k-1},\Th}\\
	\yk &= \F{\v{h}_{k}}{\xk,\v{r}_{k},\Th}.
\end{align}
\end{subequations}
Here the stochasticity is separated into the noise processes $\v{q}$ and $\v{r}$ which are usually
assumed to be zero mean, white and independent of each other. We will restrict ourselves
to the case of zero mean, white and additive Gaussian noise. Furthermore, 
the dynamic, measurement and both noise processes will be assumed \emph{stationary}.
This means that $\f_k$ and $\h_k$ and the PDFs of $\v{q}_{k-1}$ and $\v{r}_k$ will be independent
of $k$. Thus the SSMs considered in this thesis are of the form
\begin{subequations}
\label{eq:ssm_general}
\begin{alignat}{2}
	\xk &= \ff+\v{q}_{k-1}, \qquad& \v{q}_{k-1}&\sim \N{\v{0}}{\QQ}{\big} \label{eq:Q}\\
	\yk &= \hh+\v{r}_{k}, & \v{r}_k&\sim \N{\v{0}}{\RR}{\big} \label{eq:R}\\
	\v{x}_{0} &\sim \N{\muu}{\Sig}{\big}\label{eq:prior}.
\end{alignat}
\end{subequations}
Regarding the Gaussian probability distribution,
suppose $\x$ is normally distributed with mean $\m$ and covariance matrix $\P$.
We will then use the notation $\x \sim \N{\m}{\P}$ for ``distributed as'', 
whereas ``distribution of'' is denoted as $\Pdf{\x}=\N[\x]{\m}{\P}$, where
the Gaussian probability density function is
\begin{align}
	\N[\x]{\m}{\P}&\equiv \detr{2\pi\P}^{-\sfrac{1}{2}}
	\exp\fparen[\big]{-\sfrac{1}{2}\,(\x-\m)^\tr\P^{-1}(\x-\m)}.		
\end{align}
%
Clearly the mappings $\f:\mathcal{X}\to\mathcal{X}$ and
$\h:\mathcal{X}\to\mathcal{Y}$ in Equation~\eqref{eq:ssm_general} 
specify the means of the dynamic and the measurement models:
\begin{subequations}
\label{eq:ssm_distr_general}
\begin{align}
	\Pdf{\xk}{\xkk,\Th}&=\N[\xk][\big]{\ff}{\QQ}\label{eq:dynamics_distr_general}\\
	\Pdf{\yk}{\xk,\Th}&=\N[\yk][\big]{\hh}{\RR}\label{eq:measurement_distr_general}.
\end{align}
\end{subequations}
Going further, for the sake of notational clarity, we will sometimes make
the dependence on $\Th$ implicit and use the shorthand notation
\begin{align}
	\begin{aligned}
	\f_{k-1} &\equiv \ff, &
	\h_{k} &\equiv \hh\\
	\QQ* &\equiv \QQ, &
	\RR* &\equiv \RR\\
	\muu* &\equiv \muu, &
	\Sig* &\equiv \Sig.
	\end{aligned}
	\label{eq:shorthand_theta}
\end{align} 


\subsection{Bayesian optimal filtering and smoothing}

State inference
can be divided into subcategories based on the temporal relationship between the state
and the observations \parencite[see, e.g.,][]{Sarkka2006,Anderson1979}:
\begin{description}
\addtolength{\leftskip}{1cm}
	\item[Predictive distribution]
	$\Pdf{\v{x}_{k}}{\y_{0:k-1},\Th}$ is the predicted distribution of the state in the next timestep (or more generally at
	timestep $k+h$, where $h>0$) given the previous measurements.
	\item[Filtering distribution] $\Pdf{\v{x}_k}{\y_{0:k},\Th}$ is the marginal posterior distribution
	of any state $\xk$ given the measurements up to and including $\yk$.
	\item[Smoothing distribution]
	$\Pdf{\v{x}_k}{\y_{0:T},\Th}$ is the marginal posterior distribution
	of any state $\xk$, $k=1,\dots,T$, given the measurements up to and including $\y_T$.
\end{description} 


\subsubsection*{Predictive distribution}
Let us then derive a recursive formulation for computing the filtering distribution at time $k$. Let
$\Pdf{\xkk}{\y_{1:k-1}}$ be the filtering distribution of the previous step. Then 
\begin{align}
	\Pdf{\xk}{\y_{0:k-1},\Th}&=\defint{}{}{\Pdf{\xk,\xkk}{\y_{0:k-1},\Th}}{\xkk} \nonumber\\
	&=\defint{}{}{\Pdf{\xk}{\xkk}\Pdf{\xkk}{\y_{0:k-1},\Th}}{\xkk},
	\label{eq:pred_bayes}
\end{align}
which is known as the \emph{Chapman-Kolmogorov equation} \parencite[see,e.g.,][]{Sarkka2006}.
In this thesis the predictive distributions will be Gaussian or approximated with a Gaussian
\begin{align}
	\Pdf{\xk}{\y_{0:k-1},\Th}&\approx\N[\xk]{\m_{k|k-1}}{\P_{k|k-1}}.
\end{align}

\subsubsection*{Filtering distribution}
Incorporating the newest measurement can be achieved with the Bayes'
rule \parencite[see, e.g.,][]{gelman2004}
\begin{align}
	\underbrace{\Pdf{\xk}{\y_{0:k},\Th}}_\text{posterior}&=\frac{\overbrace{\Pdf{\yk}{\xk,\Th}}^\text{likelihood}\overbrace{\Pdf{\xk}{\y_{0:k-1},\Th}}^\text{prior}}{\underbrace{\Pdf{\y_{k}}{\y_{0:k-1},\Th}}_\text{normalization
	constant}}\nonumber\\
	&=\frac{\Pdf{\yk}{\xk}\Pdf{\xk}{\y_{0:k-1}}}{\defint{}{}{\Pdf{\yk}{\xk}\Pdf{\xk}{\y_{0:k-1}}}{\xk}},
	\label{eq:filt_bayes}
\end{align}
which is called the measurement update equation.
In this thesis the filtering distributions will be Gaussian or approximated with a Gaussian
\begin{align}
	\Pdf{\xk}{\y_{0:k},\Th}&\approx\N[\xk]{\m_{k|k}}{\P_{k|k}}.
\end{align}


\subsubsection*{Smoothing distribution}
The smoothing distributions can also be computed recursively by assuming that the filtering distributions
and the smoothing distribution $\Pdf{\x_{k+1}}{\Y}$ of the ``previous'' step are available.
Since
\begin{align*}
	\Pdf{\xk}{\x_{k+1},\Y,\Th}&=\Pdf{\xk}{\x_{k+1},\y_{0:k},\Th}\\
	&=\frac{\Pdf{\xk,\x_{k+1}}{\y_{0:k},\Th}}{\Pdf{\x_{k+1}}{\y_{0:k},\Th}}\\
	&=\frac{\Pdf{\x_{k+1}}{\xk,\Th}\Pdf{\xk}{\y_{0:k},\Th}}{\Pdf{\x_{k+1}}{\y_{0:k},\Th}}
\end{align*}
we get
\begin{align}
	\Pdf{\xk,\x_{k+1}}{\Y,\Th}&=\overbrace{\Pdf{\xk}{\y_{0:k},\Th}}^\text{filtering}\frac{\overbrace{\Pdf{\x_{k+1}}{\xk,\Th}}^\text{dynamic}\Pdf{\x_{k+1}}{\y_{0:T},\Th}}{\underbrace{\Pdf{\x_{k+1}}{\y_{0:k},\Th}}_\text{predictive}},
	\label{eq:smooth_joint_bayes}
\end{align}
so that the marginal is given by
\begin{align}
	\Pdf{\xk}{\Y,\Th}&=\Pdf{\xk}{\y_{0:k},\Th}\defint{}{}{\left[\frac{\Pdf{\x_{k+1}}{\xk,\Th}\Pdf{\x_{k+1}}{\Y,\Th}}{\Pdf{\x_{k+1}}{\y_{0:k},\Th}}\right]}{\x_{k+1}},
	\label{eq:smooth_bayes}
\end{align}
where $\Pdf{\x_{k+1}}{\y_{0:k}}$ can be computed by Equation~\eqref{eq:pred_bayes}.
In this thesis the smoothing distributions will be Gaussian or approximated with a Gaussian
\begin{align}
	\Pdf{\xk}{\Y}&\approx\N[\xk]{\m_{k|T}}{\P_{k|T}}.
	\label{eq:smoothing_gaussian}
\end{align}

\subsubsection*{Marginal likelihood}

An important quantity concerning parameter estimation is the marginal likelihood $\Pdf{\Y}{\Th}$. 
If we're able to compute the distributions
\begin{align}
	\Pdf{\y_k}{\y_{0:k-1},\Th}&=\defint{}{}{\Pdf{\yk}{\xk,\Th}\Pdf{\xk}{\y_{0:k-1},\Th}}{\xk},
	\label{eq:y_pred}
\end{align}
which we recognize as the ``normalization constant'' in \eqref{eq:filt_bayes},
then by repeatedly applying the definition of conditional probability 
we find that the marginal likelihood can be computed from
\begin{align}
	\Pdf{\Y}{\Th}&=\Pdf{\y_0}{\Th}\prod_{k=1}^T \Pdf{\y_k}{\y_{0:k-1},\Th}.
	\label{eq:lh_factorization}
\end{align}
Since \eqref{eq:y_pred} is needed for the filtering distributions, the marginal likelihood, or an approximation to it, can be easily
computed with the chosen filtering algorithm. Equation~\eqref{eq:lh_factorization} is sometimes known 
as the \emph{prediction error decomposition} \parencite{Harvey1990}.




