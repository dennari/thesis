\subsection{State space models}

State space models (SSMs) provide a unified probabilistic methodology for modeling
sequential data \parencite{ljung1994modeling,durbin2012time,Cappe2005,Barber2011}. Sequential data arise in numerous applications, typically in the form
of time-series measurements \todo{examples}. However it is not necessary for the sequence index to have
a temporal meaning. In this thesis we will denote the indexing variable as $t$ and it
is assumed to be discrete and nonnegative. We will use the shorthand notation $\v{x}_k$ for the 
the value of the stochastic process at time $t_k$, i.e. of $X(t_k)$. Also the shorthand $\v{x}_{1:k}$
will be used to mean $\{\v{x}_1,\dots,\v{x}_k\}$.

A fundamental question in probabilistic models for sequential data is how 
to model the dependence between variables. It is infeasible to assume
that every random variable in the process depends on all the others.
Thus it is common to assume a \emph{Markov chain}, where the value of
the process at the current timestep depends only on the previous one.
A further assumption in SSMs is that the process is not directly observed
but only through another stochastic process, the \emph{measurement process}.
The random variables of the measurement process are conditionally independent
given the latent Markov process.
An intuitive way to present conditional independence properties between random
variables is a \emph{Bayes network} presented by directed acyclic graph (DAG) \parencite{pearl1988probabilistic,Bishop2006}.
A Bayes network presentation of a SSM is given in figure~\ref{fig:ssm_graphical}.

\begin{figure}[!htp]
	\centering
	\begin{tikzpicture}
	\tikzstyle{main}=[circle, minimum size = 5mm, inner sep=0mm, thick, draw =black!80, node distance = 12mm,font=\small]
	\tikzstyle{param}=[circle, minimum size = 2mm, inner sep=0mm, thick, draw =white!80, fill=black!80, node distance = 18mm]
	\tikzstyle{ellipsis}=[circle, minimum size = 10mm, thick, draw =black!80, node distance = 18mm]
	\tikzstyle{connect}=[-latex, semithick]
	  %\node[main,node distance=50mm] (theta) [label=above:$\theta$] {};
	  \node[main] (x_1) [label=above:$\v{x}_{k-1}$] {};
	  \node[main,draw=white!0] (prev) [left of=x_1] {$\dots$};
	  \node[main] (x) [right of=x_1,label=above:$\v{x}_{k}$] {};
	  \node[main] (x__1) [right of=x,label=above:$\v{x}_{k+1}$] {};
	  \node[main,fill = black!10] (y_1) [below of=x_1,label=below:$\v{y}_{k-1}$] {};
	  \node[main,fill = black!10] (y) [below of=x,label=below:$\v{y}_{k}$] {};
	  \node[main,fill = black!10] (y__1) [below of=x__1,label=below:$\v{y}_{k+1}$] {};
	  \node[main,draw=white!0] (next) [right of=x__1] {$\dots$};
	
	  \path
	    (prev) edge [connect] (x_1)
	    (x_1) edge [connect] (x) 
	    	  edge [connect] (y_1)
	    (x) edge [connect] (y)
	    	edge [connect] (x__1)
	    (x__1) edge [connect] (y__1)
	    	edge [connect] (next);
	\end{tikzpicture}
	\caption{SSM as a graphical model presented with a directed acyclic graph}
	\label{fig:ssm_graphical}
\end{figure}
Taking into account the Markov property of the state process and the conditional
indepence property measurement process, the joint distribution of states
and measurements factorises as
\begin{align}
	\Pdf{\v{x}_{0:T},\v{y}_{1:T}}&=\Pdf{\v{x}_0}\prod_{k=1}^T\Pdf{\v{x}_k}{\v{x}_{k-1}}\Pdf{\v{y}_k}{\v{x}_{k}}
	\label{eq:joint_factorization}.
\end{align}
Thus in order to describe a SSM one needs to specify three probability distributions:
\begin{description}
\addtolength{\leftskip}{1cm}
	\item[Prior distribution]
	$\Pdf{\v{x}_0}$ is the distribution assumed for the state prior to observing any measurements. The
	sensitivity of the posterior distributions to the prior depends on the amount of data (the more data the less sensitivity).
	\item[Dynamics model]
	$\Pdf{\v{x}_k}{\v{x}_{k-1}}$ dictates the time evolution of the states
	\item[Measurement model]
	$\Pdf{\v{y}_k}{\v{x}_{k}}$ models how the observations depend on the state and the statistics of the noise
\end{description}

\subsubsection*{Example: 1D random walk}
The simplest example is a one dimensional random-walk observed in Gaussian noise.
We will assume $\Pdf{\v{x}_0}=\N{0,P_0}$. In an alternative (but equivalent) notation
the dynamics model is now
\begin{align}
	\Pdf{x_k}{x_{k-1}}&=x_{k-1}+q_{k-1},
\end{align}
where $q_{k-1}\sim \N{0,Q}$ and the measurement model is
\begin{align}
	\Pdf{y_k}{x_{k}}&=x_{k}+r_{k},
\end{align}
where $r_{k}\sim \N{0,R}$. A simulation from the model is presented in figure~\ref{fig:rw1d}.

\begin{figure}[htp]
\begin{center}
	\missingfigure{A missing 1D RW simulation}
  \caption{Simulation from the 1D RW model}
  \label{fig:rw1d}
\end{center}
\end{figure}


In a SSM it is assumed that at time $k$ the system is in \emph{state} $\v{x}_k\in\R^{d_x}$.\todo{This whole chapter needs to be refactored}\todo{stationarity} 
The state vectors are random variables which contain the quantities of
interest in the system, such as position and velocity in case of a kinetics model. The state is not observed
directly, instead at time $k$ we acquire a \emph{noisy} measurement $\v{y}_k\in\R^{d_y}$. Since the states
are not observed, they are called hidden or latent variables.\todo{explain hidden variables}.
The system state evolves according to the \emph{dynamics} equation
\begin{align}
	\v{x}_k&=\v{f}(\v{x}_{k-1})+\v{q}_{k-1} \label{eq:dynamics_nonlinear}\\
	\v{q}_{k-1} &\sim \N{0,\v{Q}} \label{eq:pdf_dynamic_noise_nonlinear},
\end{align}
where $f:\R^{d_x}\to\R^{d_x}$ and in this thesis we restrict ourselves to additive Gaussian noise. 
It is assumed that the states form a first order \emph{Markov chain},
so that the current state is conditionally independent of the earlier states given the previous state:
\begin{align}
	\Pdf{\v{x}_k}{\v{x}_{1:k-1}}=\Pdf{\v{x}_k}{\v{x}_{k-1}}
\end{align}
The observations depend on the state through the measurement equation
\begin{align}
	\v{y}_k&=\v{h}(\v{x_k})+\v{r}_k \label{eq:measurements_nonlinear}\\
	\v{r}_{k} &\sim \N{0,\v{R}} \label{eq:pdf_measurement_noise_nonlinear},
\end{align}
where $h:\R^{d_x}\to\R^{d_y}$.
An equivalent way of presenting equations \eqref{eq:dynamics_nonlinear}, \eqref{eq:pdf_dynamic_noise_nonlinear}, 
\eqref{eq:measurements_nonlinear} and \eqref{eq:pdf_measurement_noise_nonlinear}
is
\begin{align}
	\Pdf{\v{x}_k}{\v{x}_{k-1}}=\N{\v{f}(\v{x}_{k-1}),\v{Q}}\\
	\Pdf{\v{y}_k}{\v{x}_{k}}=\N{\v{h}(\v{x}_{k}),\v{R}}.
\end{align}

\subsection{Bayesian optimal filtering and smoothing}

Inference can be defined as answering questions of interest with a probability distribution \parencite{barber2011bayesian}.
In case of SSMs there are many questions of interest, but most commonly one would
like to know the \emph{marginal posterior distribution} of the states. State inference
can be divided into subcategories based on the temporal relationship between the state
and the observations \parencite{Sarkka2006}:
\begin{description}
\addtolength{\leftskip}{1cm}
	\item[Predictive distribution]
	$\Pdf{\v{x}_{k}}{\v{y}_{1:k-1}}$ is the predicted distribution of the state in the next timestep (or more generally at timestep $k+h$, where $h>0$) 
	given the previous measurements
	\item[Filtering distribution] $\Pdf{\v{x}_k}{\v{y}_{1:k}}$ is the marginal posterior distribution
	of any state $\xk$ given the measurements up to and including $\yk$
	\item[Smoothing distribution]
	$\Pdf{\v{x}_k}{\v{y}_{1:T}}$ is the marginal posterior distribution
	of any state $\xk$ given the measurements up to and including $\y_T$ where $k<T$
\end{description} 

\subsubsection*{Predictive distribution}
Let us now derive a recursive formulation for computing the filtering distribution at time $k$. Let $\Pdf{\xkk}{\y_{1:k-1}}$
be the filtering distribution of the previous step. Then 
\begin{align}
	\Pdf{\xk}{\y_{1:k-1}}&=\defint{}{}{\Pdf{\xk,\xkk}{\y_{1:k-1}}}{\xkk} \nonumber\\
	&=\defint{}{}{\Pdf{\xk}{\xkk}\Pdf{\xkk}{\y_{1:k-1}}}{\xkk},
	\label{eq:pred_bayes}
\end{align}
which is known as the \emph{Chapman-Kolmogorov equation} \parencite{Sarkka2006}.
\subsubsection*{Filtering distribution}
Incorporating the newest measurement can be achieved with the Bayes'
rule (see for example \cite{gelman2004})
\begin{align}
	\underbrace{\Pdf{\xk}{\y_{1:k}}}_\text{posterior}&=\frac{\overbrace{\Pdf{\yk}{\xk}}^\text{likelihood}\overbrace{\Pdf{\xk}{\y_{1:k-1}}}^\text{prior}}{\underbrace{\Pdf{\y_{k}}{\y_{1:k-1}}}_\text{marginal likelihood}}\nonumber\\
	&=\frac{\Pdf{\yk}{\xk}\Pdf{\xk}{\y_{1:k-1}}}{\defint{}{}{\Pdf{\yk}{\xk}\Pdf{\xk}{\y_{1:k-1}}}{\xk}}
\end{align}
which is called the measurement update equation.

\subsubsection*{Smoothing distribution}
The smoothing distributions can also be computed recursively by assuming that the filtering distributions
and the smoothing distribution $\Pdf{\x_{k+1}}{\y_{1:T}}$ of the ``previous'' step are available.
Since
\begin{align*}
	\Pdf{\xk}{\x_{k+1},\y_{1:T}}&=\Pdf{\xk}{\x_{k+1},\y_{1:k}}\\
	&=\frac{\Pdf{\xk,\x_{k+1}}{\y_{1:k}}}{\Pdf{\x_{k+1}}{\y_{1:k}}}\\
	&=\frac{\Pdf{\x_{k+1}}{\xk}\Pdf{\xk}{\y_{1:k}}}{\Pdf{\x_{k+1}}{\y_{1:k}}}
\end{align*}
we get
\begin{align}
	\Pdf{\xk}{\y_{1:T}}&=\Pdf{\xk}{\y_{1:k}}\defint{}{}{\left[\frac{\Pdf{\x_{k+1}}{\xk}\Pdf{\x_{k+1}}{\y_{1:T}}}{\Pdf{\x_{k+1}}{\y_{1:k}}}\right]}{\x_{k+1}},
\end{align}
where $\Pdf{\x_{k+1}}{\y_{1:k}}$ can be computed by equation \eqref{eq:pred_bayes}.

\subsubsection*{Marginal likelihood}

An important quantity concerning parameter estimation is the marginal likelihood $\Pdf{\y_{1:T}}$. Since
\begin{align}
	\Pdf{\y_k}{\y_{1:k-1}}&=\defint{}{}{\Pdf{\yk}{\xk}\Pdf{\xk}{\y_{1:k-1}}}{\xk}
\end{align}
the marginal likelihood can be computed from
\begin{align}
	\Pdf{\y_{1:T}}&=\Pdf{\y_1}\prod_{k=2}^T \Pdf{\y_k}{\y_{1:k-1}}
\end{align}

\subsection{Parameters in SSM}

We will assume that the prior distribution, the dynamics model and the measurement model are known
except for a set of parameters $\Th$. Now the joint
distribution of all the variables in the SSM can be written as
\begin{align}
	\Pdf{\v{x}_{0:T},\v{y}_{1:T},\Th}&=\Pdf{\Th}\Pdf{\v{x}_0}{\Th}\prod_{k=1}^T\Pdf{\v{x}_k}{\v{x}_{k-1},\Th}\Pdf{\v{y}_k}{\v{x}_{k},\Th}.
\end{align}

